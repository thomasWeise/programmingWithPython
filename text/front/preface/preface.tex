\chapter*{Preface}%
\pdfbookmark{Preface}{preface}%
%
This book tries to teach undergraduate and graduate students as well as high school students how to program with the \python\ programming language.
It aims to strike a good balance between theory and practice, leaning more to the practice side.
In particular, we try to teach programming together with some software engineering concepts.
It is the firm opinion of the author that these two cannot be separated.
Teaching programming alone without introducing tools such as static code analysis, unit tests, and enforcing principles such as code style and proper commenting will create bad programmers.
So we discuss these aspects while working our way through the principles of programming.

This book is intended to be read on an electronic device.
Please do not print it.
Help preserving the environment.

This book is work in progress.
It will take years to be completed and I plan to keep improving and extending it for quite some time.

This book is freely available.
You can download its newest version from \url{https://thomasweise.github.io/programmingWithPython/}.
This version may change since this book is, well, work in progress.
The book is released under the \emph{Attribution-NonCommercial-ShareAlike~4.0 International license} (\href{http://creativecommons.org/licenses/by-nc-sa/4.0/}{\mbox{CC~BY-NC-SA~4.0}}).
You can freely share it.
You can also copy text or figures, as long as you cite the book as the original source~\cite{programmingWithPython}, e.g., by using the following Bib\TeX:%
%
\begin{lstlisting}[style=text_style]
@book{programmingWithPython,
 author = {Thomas Weise},
 title = {Programming with Python},
 year = {2024--2025},
 publisher = {Institute of Applied Optimization,
              School of Artificial Intelligence and Big Data,
              Hefei University},
 address = {Hefei, Anhui, China},
 url = {https://thomasweise.github.io/programmingWithPython}
}
\end{lstlisting}
%
This book contains a lot of examples.
You can find all of them in the repository \url{\programmingWithPythonCodeRepo}.
You can clone this repository and play with these example codes.%
%
\begin{sloppypar}%
The text of the book itself is also available in the repository \url{\programmingWithPythonRepo}.
There, you can also submit \href{\programmingWithPythonRepo/issues}{issues}, such as change requests, suggestions, errors, typos, or you can inform me that something is unclear, so that I can improve the book.
Such feedback is most welcome.
The book is written using \LaTeX\ and this repository contains all the scripts, styles, graphics, and source files of the book (except the source files of the example programs).%
\end{sloppypar}%
%
\strut\vfill\strut%
%
\copyrightBlock{2024--2025}%
%
\clearpage%
%
\strut\vfill\strut%
%
\begin{center}%
\noindent\resizebox{\linewidth}{!}{%
\begin{tabular}{c@{~~~~~~~~}c}%
\includegraphics[width=7cm]{\currentDir/bookUrl}&\includegraphics[width=7cm]{\currentDir/courseUrl}\\\relax%
{\huge{[\expandafter\href{\programmingWithPythonUrl/programmingWithPython.pdf}{book pdf}]}}&{\huge{[\expandafter\href{\programmingWithPythonUrl}{course website}]}}\\%
\end{tabular}}%
\\[12pt]\noindent%
\resizebox{0.85\linewidth}{!}{\expandafter\url{\programmingWithPythonUrl}}%
\end{center}%
%
\strut\vfill\strut%
This book was built using the following software:%
\gitOutputWithStyle{}{}{bookbase/scripts/dependencyVersions.sh}{versions}{}{text_style}%
