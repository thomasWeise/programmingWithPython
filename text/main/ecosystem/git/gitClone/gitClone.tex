\hsection{Cloning git Repositories}%
\label{sec:gitClone}%
%
The most fundamental activity you will encounter is \emph{cloning} repositories.
Under \git, this basically means to download the complete repository and its history to your machine.
You can now make local changes to the downloaded files.
You can create commits, that will change the local version of your repository.
Then you could push these changes back to the repository hosted by the \git\ \pgls{server}, making them accessible for other users.
But, well, the first step is to clone - i.e., to download -- the repository.

You can clone \git\ repositories with the command line \bashil{git} \pgls{client} program.
However, \pycharm\ also has a \git \pgls{client} built in.
We here outline both approaches.%
%
\hinput{gitClonePycharm}{gitClonePycharm}%
\hinput{gitCloneClient}{gitCloneClient}%
%
\endhsection%
