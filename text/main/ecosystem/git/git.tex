\hsection{The Distributed Version Control System git}%
%
Today, \git~\cite{S2023LG,T2024BGAGVCPMATFTND} is maybe the \pgls{VCS} with the most wide-spread use.
It is the \pgls{VCS} on which \github~\cite{PRGWSUdVLFTEKPKFBV2016TSRFTAOGAG} is based, which, in turn, is maybe the most important hub for open source software projects in the world.
\git\ is based on a \pgls{clientServerArchitecture}, where the \pgls{server} hosts and manages repositories of source code and other resources.
The \git\ \pgls{client} is a command line application that is run in the \pgls{terminal} and which allows you to clone (i.e., download) source code repositories and upload (i.e., commit) changes to them.
A repository is something like a directory with files and their editing history, i.e., you can work and improve source code, commit changes, and see the history of all past commits.
This is what \pglspl{VCS} are for:
They do not just provide the current state of a project and allow teams to cooperative and continuous develop software, they also store the history of the project so as to enable us to see which code was used in which version of our software and to track changes.
How to correctly use \git\ thus is a very complex and involved topic beyond the scope of this book.
However, we will here take at least a look into a very small subset of functions that provide you a starting point for working with \git\ repositories.%
%
\hinput{gitClone}{gitClone.tex}%
%
\endhsection%
