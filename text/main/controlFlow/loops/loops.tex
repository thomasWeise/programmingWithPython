\hsection{Loops}%
%
When we are working with sequences of data, we do not just want to perform an action on one data element.
We instead often want to apply the actions repetitively to all data elements.
Loops allow us to do just that, to perform the same actions multiple times.%
\pythonIdx{loop}%
%
\hsection{The \texttt{for} Loop Statement}%
%
\pythonIdx{loop!for}%
The most basic such sequence in \python\ may be the \pythonilIdx{for}~loop, which has the following pattern:%
%
\begin{pythonSyntax}
for loopVariable in sequence:
    loop body statement 1
    loop body statement 2
    # ...

normal statement 1
normal statement 2
# ...
\end{pythonSyntax}
%
%
\gitPythonAndOutput{\programmingWithPythonCodeRepo}{04_loops}{for_loop_range.py}{--args format}{loops:for_loop_range}{%
An example for using the \pythonilIdx{for} loop over a \pythonilsIdx{range} of integer numbers.}%
%
\gitPythonAndOutput{\programmingWithPythonCodeRepo}{04_loops}{for_loop_pi_liu_hui.py}{--args format}{loops:for_loop_pi_liu_hui}{%
A variant of \cref{lst:variables:pi_liu_hui} which uses a \pythonilIdx{for} loop instead of five copies of the same instructions.}%
%
\gitPythonAndOutput{\programmingWithPythonCodeRepo}{04_loops}{for_loop_continue_break.py}{--args format}{loops:for_loop_continue_break}{%
An example for the \pythonilIdx{continue} and \pythonilIdx{break} statements in a \pythonilIdx{for} loop.}%
%
The keyword~\pythonilIdx{for} is followed by a loop variable.
Then comes the keyword~\pythonilIdx{in}, the \pythonil{sequence} we want to iterate over, and finally a colon~(\pythonilIdx{:}).
This variable will iteratively take on the values in the \pythonil{sequence}.
The loop body statements in the following, indented block are executed for each of these values.
After the loop, we leave a blank line followed by the code that will be executed after the loop completes.

In its most simple form, the \pythonilIdx{for} loop is applied to a \pythonilIdx{range} of integer numbers.
Ranges are sequences which work basically like slices\pythonIdx{slice} (see \cref{sec:lists:basicFunctions,sec:strBasicOperations}).
\pythonil{range(5)} will give us a sequence of integers starting with~0 and reaching up to right \emph{before}~5, i.e., the integer range~\intRange{0}{4}.
\pythonil{range(6, 9)} gives the sequence of integers starting with~6 and stopping right \emph{before}~9, i.e., the integer range~\intRange{6}{8}.
Finally, \pythonil{range(20, 27, 2)} results in a sequence of integers that begins at~20, increments by~2 in each step, and ends right before~27.
This is the sequence~$(20, 22, 24, 26)$.
\pythonilsIdx{range}, like slices\pythonIdx{slice}, can also have negative increments:
The \pythonil{range(40, 30, -3)} starts with~40 and stops before reaching~30 and decrements by~3 in each step.
This is equivalent to the set~$(40, 37, 34, 31)$.

In \cref{lst:loops:for_loop_range}, we loop over exactly these ranges.
In this listing, we try to create a dictionary (see \cref{sec:dictionaries}) where some integer numbers are mapped to their squares.
We use four \pythonilIdx{for} loops to fill this dictionary with data.%
%
\endhsection%
%
\endhsection%
%
