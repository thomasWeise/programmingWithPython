\hsection{Control Flow Statements}%
%
With the things we learned so far, we can write simple linear programs.
These programs perform instruction after instruction in exactly the same order in which we write them down.
If we want something done twice, we have to write it down twice.
Our programs have no way to adapt their behavior based on their input data.
They cannot do one thing if a variable has a certain value and do something else otherwise.
They do not branch or loop.
Now we will learn how to write programs that \emph{do} branch or loop.
We will learn statements that change the \emph{control flow}.%
%
\begin{definition}[Control Flow]%
The \emph{control flow} is the order in which the statements in a program are executed.%
\end{definition}%
%
\hinput{conditionals}{conditionals.tex}%
\hinput{loops}{loops.tex}%
%
\endhsection%
%
