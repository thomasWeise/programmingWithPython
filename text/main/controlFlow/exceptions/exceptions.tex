\hsection{Exceptions}%
\label{sec:exceptions}%
%
So far, we have mainly focused on writing correct code.
We try to create code that is free of errors.

When we execute our programs, then there are at least two things that can go wrong.
On one hand, we can never really be sure that our program code is free of programming mistakes.
The larger a project gets, the more likely it is that there are some bugs hidden somewhere in the code.
Thorough unit testing can reduce the likelihood of bugs, but it cannot entirely prevent them.

On the other hand, our programs do not exist all by themselves in a vacuum.
They receive input, maybe from files, maybe from the user, maybe from sensors.
This input may be wrong.

Coarsely, we can group both kinds of problems together as situations that were not anticipated, that are exceptions from the intended program flow.
We already have encountered such situations.
For example, trying to access a character of a string at an index greater than or equal to the length of the string will lead to an~\pythonilIdx{IndexError}, as we saw in \cref{sec:strBasicOperations}.
The attempt to modify an element of a \pythonil{tuple} is punished with a \pythonilIdx{TypeError} in \cref{sec:tuples}.
Back in \cref{sec:float:special}, we saw that trying to compute something like \pythonil{(10 ** 400) * 1.0} will yield an \pythonilIdx{OverflowError}, as the integer~$10^{400}$ is too large to be converted to a \pythonil{float} during the multiplication with~\pythonil{1.0}.

Clearly, some of these errors may result from programming mistakes.
But they could just as well result from invalid data being entered in the input of the program.%
%
\hsection{Introduction}%
When we create a new function or program, we have to face the question:
\emph{What should we do if we receive incorrect input?}
We can imagine three different approaches:%
%
\begin{enumerate}%
%
\item We simply ignore the issue.
If the input of our program or function is faulty, then the output will be wrong, too.
This is often called \pgls{GIGO}~\cite{PKBCBR2005GIGOAELAOMBEUP}.
Converting the integer \pythonil{10 ** 400} to a \pythonil{float} could just yield~\pythonil{inf}, for example.%
%
\item We try to sanitize the input.
For example, our \pythonil{factorial} function in \cref{lst:functions:def_factorial} expects an integer as input.
If someone were to pass in the floating point value~\pythonil{2.4} instead, we could round it to~\pythonil{2} and return the result.
Matter of fact, our \pythonil{sqrt} function implemented in \cref{lst:functions:my_math_3} returns \pythonil{0.0} if we pass in a negative number.%
%
\item We can guard the function by raising an \pythonilIdx{Exception}~\cite{MAGTOC2024EHFAP,SF2013ASROSR,GRRX2001ACSOEHMFBDOOS}.%
%
\end{enumerate}%
%
The latter is what \python\ does in the examples mentioned initially:
While it could simply ignore if we try to overwrite an element of a \pythonil{tuple}, it instead raises an~\pythonilIdx{TypeError}, for example.
Personally, I am also a fan of this approach.

If we would follow the \pgls{GIGO} paradigm, then faulty data will propagate.
Maybe the output of our function is fed as input into another function, whose output is then piped into another function, and so on.
An error could then will lead to some crash down the line.
If functions that are written under the assumption that \pgls{GIGO} is OK are paired with such that perform input sanitization, errors could remain unnoticed.
The erroneous results could then become part of some actual, real-life decisions and designs.
And even if found out, it will be extremely hard to discover where things went wrong in the long chain of computations and function calls.

Input sanitization could cover some error that happened earlier.
This could even cause more errors, because it will allow other programmers to call our code with wrong values.
Input sanitization encourages sloppy programming.

This leaves raising an \pythonilIdx{Exception}.
But what does that actually mean?
Raising an \pythonilIdx{Exception} means two things:%
%
\begin{enumerate}%
%
\item We store information about the error and information about the current execution state (current line of code and the function call hierarchy) in an object (the \pythonilIdx{Exception}).%
%
\item The control flow immediately leaves the currently executed block of instructions as well as all calling blocks or functions.
It jumps up in the call hierarchy until reaching code that handles the raised \pythonilIdx{Exception} is reached.
If no such code exists, the current process is terminated with an \pgls{exitCode} different from~\textil{0}.%
%
\end{enumerate}%
%
In other words, raising an \pythonilIdx{Exception} is a way to exit the current control flow and to signal an error that must either explicitly be handled by code or will lead the process to terminate.
In my opinion, this is the best way to handle incorrect input or other erroneous situations for several reasons:%
%
\begin{enumerate}%
%
\item It clearly and explicitly shows that an error has happened, \emph{where} it happened, \emph{when} it happened, and, to some degree, \emph{why} it happened.
This makes it much easier to find out whether the error is caused by invalid input or by a programming mistake.%
%
\item It prevents \pgls{GIGO} from occurring.
Thus, it prevents a faulty situation or corrupted data from propagating out of the current context.
If an error occurs, this the raised \pythonilIdx{Exception} takes down the current path of execution and this stops the contaminated control flow.%
%
\item It forces programmers to explicitly deal with the error condition.
An error cannot be simply ignored.
Indeed, someone who calls our code might write code that ignores or discards the \pythonilsIdx{Exception} that we raised.
But they have to do so \emph{explicitly} in their code.
Thus, they have to \emph{intentionally} deal with the possible error condition.
It cannot happen that an error gets overlooked.
An \pythonilIdx{Exception} that is not handled will lead to the termination of the process.%
%
\item One might argue: \inQuotes{But what if the process crashes because an \pythonilIdx{Exception} is not handled?}
The answer may be:
What is worse?
That an error causes the current process to crash unexpectedly or that all future results after the error are wrong and, even worse, are wrong \emph{unnoticed}?%
%
\end{enumerate}%
%
\bestPractice{exceptions}{%
Errors should not be ignored and input should be sanitized. %
Instead, the input of our functions should be checked for validity (where reasonable). %
\pythonilsIdx{Exception} should be raised as early as possible if any unexpected situation occurs.%
}%
%
\endhsection%
%
\hsection{Raising Exceptions}%
%
\gitPython{\programmingWithPythonCodeRepo}{06_exceptions/sqrt_raise.py}{--args format}{exceptions:sqrt_raise}{%
A new variant of the \pythonil{sqrt} function from back in \cref{lst:functions:my_math_3} that raises an \pythonilIdx{ArithmeticError} if its input is non-finite or negative.}%
%
\gitPythonAndErrorOutput{\programmingWithPythonCodeRepo}{06_exceptions}{use_sqrt_raise.py}{--args format}{exceptions:use_sqrt_raise}{%
Using the new variant of the \pythonil{sqrt} function from \cref{lst:exceptions:sqrt_raise} that raises an \pythonilIdx{ArithmeticError} if its input is non-finite or negative.}%
%
Let us now first look at an example of how we can signal an error condition in our code.
For this purpose, we re-visit the square root function that \pythonil{sqrt} we implemented back in \cref{lst:functions:my_math_3} in \cref{sec:unitTesting}.
In that implementation, we realized that certain input values such as \pythonil{inf}, \pythonil{-inf}, \pythonil{nan}, and \pythonil{0.0} deserve special treatment.
We also found that nothing could stop a user to pass a negative number as input to our \pythonil{sqrt} implementation.
We did not yet have any means to deal with nonsense in a reasonable way, so we decided to just return~\pythonil{0.0} in that case.

Obviously, this is a bad idea because passing a negative number to \pythonil{sqrt} can only mean two things:
Either the programmer who did that does not know that a square root is.
Or the negative input came as the result from another computation and that computation somehow was wrong.
In the first case, we should somehow make it explicit to the programmer that the square root of a negative number is not defined and that they should question their approach to mathematics.
In the second case, we should rather stop the computation right there and then before the incorrect results propagate and do some damage elsewhere.
In both cases, returning~\pythonil{0.0} is not a good idea.

Instead, we want to signal this error explicitly.
Our function should raise an \pythonilIdx{Exception}.
We therefore create a new implementation of \pythonil{sqrt} in \cref{lst:exceptions:sqrt_raise}.
Before writing the actual \python\ code, we will express the new behavior in the \pgls{docstring} of our function:%
%
\bestPractice{exceptionDoc}{%
\sloppy%
Any function that may raise an exception should explain any exception that it explicitly raises in the \pgls{docstring}. %
This is done by writing something like \textil{:raises ExceptionType: why} where \textil{ExceptionType} is to be replaced with the type of the exception raised and \textil{why} with a brief explanation why it will be raised.%
}%
\fussy%
%
In our case, any input~$x$ for which~$\sqrt{x}$ would be either undefined or not finite should lead to an error.
Since this error is of arithmetic nature, we will raise an~\pythonilIdx{ArithmeticError}.
Our \pgls{docstring} therefore contains the line \textil{:raises ArithmeticError: if `number` is not finite or less than `0.0`}.
Any other programmer using our code therefore can easily see what kind of \pythonilsIdx{Exception} our code could raise.

In the new \pythonil{sqrt} implementation \cref{lst:exceptions:sqrt_raise} we first check whether either \pythonil{not isfinite(number)}\pythonIdx{isfinite} or \pythonil{number < 0.0} holds.
The \pythonilIdx{isfinite} function is offered by the \pythonilIdx{math} module and returns \pythonil{True} if its argument is either \pythonilIdx{inf}, \pythonilIdx{-inf}, or~\pythonilIdx{nan}.
The complete expression becomes \pythonil{True} if the input of the function is not finite or negative.
We thus check whether the output of the \pythonil{sqrt} function would not be a finite number.
If this is indeed the case, we \pythonil{raise ArithmeticError(f"sqrt(\{number\}) is not permitted.")}.

\pythonilIdx{raise} is the keyword used to, well, raise an \pythonilIdx{Exception}, i.e., to signal an error.
\pythonilIdx{ArithmeticError} then creates the object with the information about the exception.
We can pass a string as parameter to this function, and we here chose to pass in an \pgls{fstring} which contains the value of the number with which our \pythonil{sqrt} function was called.
This line of code will force the control flow to immediately exit our \pythonil{sqrt} function.
The \pythonilIdx{Exception} object \inQuotes{raises} up until it is either caught (which we discuss later) or the process itself is terminated.

The latter can be observed in \cref{lst:exceptions:use_sqrt_raise}.
In the program, we iteratively apply our new \pythonil{sqrt} function to the values inside a \pythonil{tuple} using a \pythonilIdx{for}~loop.
We write the results of these computations to the \pgls{stdout} using \pgls{fstring} and \pythonil{print}.\footnote{%
The \pythonil{flush=True} parameter passed to \pythonilIdx{print}\pythonIdx{print!flush} forces the output to be immediately written and not cached. %
This is only useful to keep the output in order in this example, because the \pgls{stdout} and \pgls{stderr} are captured and presented in \cref{exec:exceptions:use_sqrt_raise} and the order of the text could otherwise get mixed up.%
}
The first five numbers are fine and \pythonil{sqrt} returns the proper results.
However, the latter three numbers, \pythonilIdx{inf}, \pythonilIdx{nan}, and \pythonil{-1.0} all would cause an error.

We collect the output of the program in \cref{exec:exceptions:use_sqrt_raise}.
As can be seen, for \pythonil{0.0}, \pythonil{1.0}, \pythonil{2.0}, \pythonil{4.0}, and \pythonil{10.0}, the results are printed as anticipated.
However, when the \pythonilIdx{for}~loop reaches \pythonilIdx{inf}, the program is terminated an the so-called \pgls{stackTrace} is written to the output.

It begins with the line \textil{Traceback (most recent call last):}.
In the first line following this text, the source code file and the index of the line in that file where the \pythonilIdx{Exception} was originally raised are printed.
The full path of the file may look a bit strange, since this book is compiled by a \github\ Action~\cite{C2024GA}, but at least the last part of the path, \textil{06_exceptions/use_sqrt_raise.py} clearly points out the calling program as the culprit \cref{lst:exceptions:use_sqrt_raise}.
Indeed, the following line of text identifies that instruction in \cref{lst:exceptions:use_sqrt_raise} that caused the error and even marks the offending function invocation by underlining it with \textil{{^}{^}{^}{^}{^}{^}{^}{^}{^}{^}{^}{^}}.
Below that, we get to see the context of our \pythonil{sqrt} function:
First, the path to its module is given (ending in \textil{06_exceptions/sqrt_raise.py}) and it is pointed out that the \pythonilIdx{Exception} was raised in line~15.
This line of code is then also given, and it indeed is the one starting with \pythonil{raise ArithmeticError}.
The \pgls{stackTrace} therefore shows us exactly where the error happened and from where the code causing the error was called.%
%
\begin{sloppypar}%
After the \pgls{stackTrace}, we can see information about the error printed that we passed in:
\textil{ArithmeticError: sqrt(inf) is not permitted.}
We made this message by ourselves using an \pgls{fstring} when raising the \pythonilIdx{Exception}.
We did this so that it tells the user that \pythonil{sqrt} was called with the argument \pythonil{inf} that we did not permit.%
\end{sloppypar}%
%
The above information allows us to pretty much identify the source of the problem.
It shall be stated here that new programmers often ignore the \pgls{stackTrace}.
They see that a program produces and error and then try to figure out why by looking at their code.
They often do not read the \pgls{stackTrace} or the error information below it.%
%
\bestPractice{exceptionStackTrace}{%
The \pgls{stackTrace} and error information printed on the \python\ console in case of an uncaught \pythonilIdx{Exception} are essential information to identify the problem. %
They should \emph{always} be read and understood before trying to improve the code.%
}%
%
In our original \pythonil{tuple} of inputs that we iteratively passed to \pythonil{sqrt}, the last three elements are \pythonilIdx{inf}, \pythonilIdx{nan}, and \pythonil{-1.0}.
The call to \pythonil{sqrt} with \pythonil{inf} as argument was performed and failed.
After that, no further output has been generated.
Indeed, the control flow has left the \pythonil{for}~loop and the process has been terminated with \pgls{exitCode}~1, as the output in \cref{exec:exceptions:use_sqrt_raise} shows.

Terminating the process may seem rash, but it is not.
If a programmer used our \pythonil{sqrt} function incorrectly, then this will force them to fix their error.
If the input \pythonil{inf} was the result of corrupted data, another erroneous computation, or an input mistake by the user, then terminating the program prevented this error from propagating.
For both scenarios, the \pgls{stackTrace} and error output gives clear information about what went wrong and where.%
%
\endhsection%
%
\hsection{Handling Exceptions}%
%
Of course, we do not want that all possible unexpected error conditions will immediately terminate our process.
For example, maybe we programmed a program for painting a picture, the user painted a picture and wants to store it, but enters a wrong destination.
It would be annoying if the program would immediately crash.%
%
\hsection{The \texttt{try}\idxdots\texttt{except} Block}%
%
Therefore, the \pythonilIdx{try}\nobreakdashes-\pythonilIdx{except} clause exists.
We place the code that may raise an \pythonilIdx{Exception} into an indented block that is prefixed by \pythonil{try:}.
After this block, we write the handlers for specific exception types.
For example, if we assume that a block could raise an \pythonilIdx{ArithmeticError}, we could write \pythonil{except ArithmeticError as ae:}\pythonIdx{as}.
The code inside this \pythonilIdx{except} block would be executed if and only if indeed an \pythonilIdx{ArithmeticError} was raised somewhere in the \pythonilIdx{try} block.
In this case, the \pythonilIdx{ArithmeticError} would be available as local variable \pythonil{ae} in this block.
Of course, multiple different types of \pythonilsIdx{Exception} may be raised, so we can have multiple \pythonilIdx{except} blocks.
This looks like this:%
%
\begin{pythonSyntax}
try:  # Begin the try-except block.
    # Code that may raise an exception or that
    # calls a function that may rise one.
except ExceptionType1 as ex1:  # One exception type that can be caught.
    # Code the handles exceptions of type ExceptionType1.
except ExceptionType2 as ex2:  # Another exception type.
    # Code the handles exceptions of type ExceptionType2 that are not
    # instances of ExceptionType1.

next statement  # The code executed after the try-except block.
\end{pythonSyntax}
%
\gitPythonAndOutput{\programmingWithPythonCodeRepo}{06_exceptions}{try_except_str_index.py}{--args format}{exceptions:try_except_str_index}{%
The \pythonil{index}\pythonIdx{str!index} function of a string raises a \pythonilIdx{ValueError} if it cannot find the given substring. %
Here we catch this error in a \pythonilIdx{try}\idxdots\pythonilIdx{except} block.}%
%
Let us now try the \pythonil{try}-\pythonil{except} block.
Back in \cref{sec:strBasicOperations}, we learned that text strings offer the method \pythonil{r.find(s)}\pythonIdx{str!find}\pythonIdx{find} that searches a string~\pythonil{s} inside~\pythonil{r} and returns the first index where it is encountered.
If \pythonil{s} cannot be found in~\pythonil{r}, \pythonil{-1}~is returned instead.
There exists a very simply operation \pythonil{r.index(s)}\pythonIdx{str!index}\pythonIdx{ind}, which works exactly like \pythonilIdx{find}, but instead of returning~\pythonil{-1}, it will raise a \pythonilIdx{ValueError} if \pythonil{s} cannot be found in~\pythonil{r}.
This is quite useful in cases where we know that \pythonil{s} must be contained in \pythonil{r} and if it is not, then that is an error.

In \cref{lst:exceptions:try_except_str_index}, we explore this behavior.
Our string~\pythonil{r} is \pythonil{"Hello World!"}.
In the \pythonilIdx{try}-block, we place a \pythonil{for} loop which lets a variable~\pythonil{s} iteratively take on two values.
In its first iteration, \pythonil{s = "Hello"} and \pythonil{r.index(s)} will yield~\pythonil{0}.
This is printed to the output.
In the second iteration, \pythonil{s = "world"}, which cannot be found since searching in strings is case-sensitive.
\pythonil{r.index(s)} will therefore raise an \pythonilIdx{Exception}.

The \pythonilIdx{except} block after the \pythonil{try} block is executed if anywhere inside the \pythonil{try} block a \pythonilIdx{ValueError} is raised.
In this case, this \pythonilIdx{ValueError} becomes available in variable~\pythonil{ve}.
In this block, we simply print the error.

After the block, we print \pythonil{"The program is now finished."}.
This code is executed only if no uncaught \pythonilIdx{Exception} has left the \pythonil{try}-\pythonil{except} block.
The output of the program given in \cref{exec:exceptions:try_except_str_index} shows that this is indeed the case:
We first get the results of the successful search for \pythonil{"Hello"}, followed by the output for the failed search.
The last line then is \textil{The program is now finished.}

\gitPythonAndErrorOutput{\programmingWithPythonCodeRepo}{06_exceptions}{try_multi_except.py}{--args format}{exceptions:try_multi_except}{%
The handling of multiple errors, namely \pythonilIdx{ZeroDivisionError} and \pythonilIdx{ArithmeticError}, as well as what happens if a variable remains unassigned due to an error (a \pythonilIdx{NameError} is raised).}%
%
In \cref{lst:exceptions:try_multi_except}, we revisit our new \pythonil{sqrt} function.
This function will \pythonilIdx{raise} an \pythonilIdx{ArithmeticError} if its argument is non-finite or negative.
This time, we want to compute~$\sqrt{\frac{1}{0}}$ and thus aim to store the result of \pythonil{sqrt(1 / 0)} in a variable \pythonil{sqrt_of_1_div_0}.
We first declare the variable as a \pythonil{float}.
Then, in a \pythonilIdx{try}-\pythonilIdx{except} block, we perform the actual computation: \pythonil{sqrt_of_1_div_0 = sqrt(1 / 0)}.
Knowing that \pythonil{sqrt} might raise an \pythonilIdx{ArithmeticError}, we provide a corresponding \pythonilIdx{except} block.
However, we also know that \pythonil{1 / 0} looks a bit dodgy, as we also try to intercept a potential \pythonilIdx{ZeroDivisionError} error.
As you can see, we can have two independent \pythonilIdx{except} clauses.

So, which one will be executed?
Certainly, $\frac{1}{0}$ is not finite, so \pythonil{sqrt} will raise an \pythonilIdx{Exception}.
Then again, $\frac{1}{0}$ cannot be computed at all, so maybe we get a \pythonilIdx{ZeroDivisionError} error instead?
We find that the \pythonilIdx{except} block for \pythonilIdx{ZeroDivisionError} is executed.
The reason is that in order to invoke \pythonil{sqrt(1 / 0)}, the \python\ interpreter must first compute the result of~\pythonil{1 / 0}.
This computation raises \pythonilIdx{ZeroDivisionError} and \pythonil{sqrt} is never called.

This leads us to the question:
If \pythonil{sqrt} is never called, then what will be assigned to \pythonil{sqrt_of_1_div_0}?
The answer is: Nothing.
In order to perform the assignment, we would need the result of \pythonil{sqrt(1 / 0)}.
But since this result never becomes available, the assignment is never performed.

This means that when we try to \pythonil{print(sqrt_of_1_div_0)} after the \pythonilIdx{try}-\pythonilIdx{except} block, the value of \pythonil{sqrt_of_1_div_0} is undefined.
The variable \pythonil{sqrt_of_1_div_0} never really received any value at all.
Trying to access it will \pythonilIdx{raise} a \pythonilIdx{NameError}.
While we did declare it, the \python\ interpreter does not know this variable yet, as it does not have a value.
Our program will terminate, because the \pythonilIdx{NameError} is never caught anywhere.
Instead, the \pgls{stackTrace} and error information will be printed in \cref{exec:exceptions:try_multi_except}.%
%
\bestPractice{noAssignAfterRaise}{%
If an \pythonilIdx{Exception} is raised, be aware that the control flow will immediately leave the current block. %
The statement in which the \pythonilIdx{Exception} was raised will not be completed but aborted right away. %
Therefore, no variable assignments or other side-effects can take place anymore and it is possible that variables remain undefined.%
}%
%
\endhsection%
%
\hsection{The \texttt{try}\idxdots\texttt{except}\idxdots\texttt{else} Block}%
\endhsection%
%
\hsection{The \texttt{try}\idxdots\texttt{finally} Block}%
\endhsection%
\endhsection%
%
%
\hsection{Testing Exceptions}%
\cite{LRBP2021AEHTPIOSL}
\endhsection%
%
\hsection{Built-in Exceptions}%
\cite{PSF2024BIE}
\endhsection%
%
\endhsection%
%
