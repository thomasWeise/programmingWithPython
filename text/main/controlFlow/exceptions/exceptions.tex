\hsection{Exceptions}%
%
So far, we have mainly focused on writing correct code.
Code that is free of errors.
At some points, however, we have come into contact with errors that may arise regardless of whether our code is correct or not.
Back in \cref{sec:float:special}, we saw that trying to compute something like \pythonil{(10 ** 400) * 1.0} will yield an \pythonilIdx{OverflowError}, as the integer~$10^{400}$ is too large to convert it to a \pythonil{float} during the multiplication with~\pythonil{1.0}.
Trying to access a character of a string at an index greater than or equal to the length of the string will lead to an~\pythonilIdx{IndexError}, as we saw in \cref{sec:strBasicOperations}.
The attempt to modify an element of a \pythonil{tuple} is rewarded with a \pythonilIdx{TypeError} in \cref{sec:tuples}.
Clearly, some of these errors may result from programming mistakes.
But they could just as well result from invalid data being entered in the input of the program.
%
\endhsection%
%
