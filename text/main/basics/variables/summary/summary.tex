\hsection{Summary}%
%
Variables allow us to store and access data.
After gaining this ability, we finally can begin to write \inQuotes{real} programs.
We do no longer just execute single commands in the \python\ console.
Instead, we can write \python\ program files (with the name suffix~\textil{.py}) that perform computations in multiple steps.
Our implementation of the ideas of LIU Hui~\cnPdf{../assignment/liuHui} to approximate the ratio of the circumference of a circle to its diameter, i.e., the number~\numberPi, \cref{lst:variables:pi_liu_hui}, was a first example.
Here, we refined the approximation in several steps and ended up with some pretty good estimate.

The elegant syntax of \python\ allows us to assign and reassign variables.
We can swap the values of two variables \pythonil{a} and \pythonil{b} by writing \pythonil{a, b = b, a}.

However, the elegant and simply syntax also has drawbacks.
It does not force us to define the type of a variable.
As a result, it does not prevent us from first storing an integer inside a variable and then storing a string into it.
This can lead to problems and confusion.
The \python\ interpreter does not care about such potential issues and will happy obey any command we give to it.
Therefore, we need additional tools:
Type checkers like \mypy\ help us to detect such potential problems.
They can tell us if we try to store something in a variable which is of a different type than the previous usage of the variable.
These programs are applied to the program files, read them, and look for errors.

These type checkers cannot guess whether the previous usage was wrong or our attempt to overwrite the value with one of a different type.
Or maybe all is good and we intended from the beginning to allow the variable to store different kinds of objects.
To allow us to create some clarity, \pglspl{typeHint} were introduced.
We can annotate variables with \pglspl{typeHint} that, basically, specify the type of objects we want to be stored in the variables.
This gives the dynamically typed \python\ language the same expressiveness of statically typed languages like \pgls{Java}.
The \python\ interpreter does not care about this and, again, happily ignores all such \pglspl{typeHint} and obeys all of our commands.
Static type checking tools, like \mypy, however, now can see our intend and provide better error messages when they detect an issue.

As final topic in this section, we tackled the issue of equality versus identity.
Two variables \pythonil{x} and \pythonil{y} are identical if they point to the same object.
Then, \pythonil{x is y} and \pythonil{y is x} will be \pythonil{True}.
Even if this is not the case, i.e., \pythonil{x is y} is \pythonil{False}, then the variables do not point to one and the same object.
However, they could still point to two objects which have the same value.
Nothing can prevent me from storing \pythonil{"Hello!"} multiple times at multiple different locations in memory.
I can let \pythonil{x} and \pythonil{y} point to two different locations both storing the same string \pythonil{"Hello!"}.
Then, \pythonil{x is y} may be \pythonil{False}, but \pythonil{x == y} could be \pythonil{True}.

Having discussed all of these issues, we can not depart from the interesting topic of variables.%
%
\endhsection%
%
