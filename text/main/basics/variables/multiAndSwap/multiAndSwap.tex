\hsection{Multiple Assignments and Value Swapping}%
%
\gitPythonAndOutput{\programmingWithPythonCodeRepo}{variables}{multi_and_swap.py}{--args format}{variables:multi_and_swap}{%
A \python\ program assigning multiple values to multiple variables and using the same method to swap variable values.}%
%
Let us return to the topic of variables and assignments.
In \python, we can assign values to multiple variables at once.
In this case, we separate both the variable names and the values with commas.
The first line (\pythonil{a, b = 5, 10}\pythonIdx{=!multiple}\pythonIdx{,}) in \cref{lst:variables:multi_and_swap} assigns the values~\pythonil{5} and~\pythonil{10}, respectively, to the variables~\pythonil{a} and~\pythonil{b}, respectively.
After this assignment step, \pythonil{a == 5} and \pythonil{b == 10} holds.
\pythonil{print(f"a={a}, b={b}")} therefore prints \textit{a=5, b=10}.

This method can also be used to \emph{swap} the values of two variables\pythonIdx{=!swap}\pythonIdx{swap}.
Writing \pythonil{a, b = b, a} looks a bit strange but it basically means \inQuotes{the \emph{new} value of \pythonil{a} will be the \emph{present} value of \pythonil{b} and the \emph{new} value of \pythonil{b} will be the \emph{present} value of \pythonil{a}.}
The line is therefore basically equivalent to first storing~\pythonil{a} in a temporary variable~\pythonil{t}, then overwriting \pythonil{a} with \pythonil{b}, and finally copying the value of \pythonil{t} to \pythonil{t}.
But it accomplishes this in single line of code instead of three.
\pythonil{print(f"a={a}, b={b}")} thus now prints \textit{a=10, b=5}.%
%
\bestPractice{swap}{Swapping of variable values can best be done with a multi-assignment statement, e.g., \pythonil{a, b = b, a}\pythonIdx{=!swap}\pythonIdx{swap}.}%
%
The same concept of multiple assignments works for arbitrarily many variables.
\pythonil{z, y, x = 1, 2, 3} assigns, respectively, \pythonil{1} to \pythonil{z}, \pythonil{2} to \pythonil{y}, and \pythonil{3} to \pythonil{x}.
\pythonil{print(f"x={x}, y={y}, z={z}")} thus yields \textil{x=3, y=2, z=1}.

We can also swap multiple values.
\pythonil{x, y, z = z, y, x} assigns the present value of \pythonil{z} to become the new value of \pythonil{x}, the present value of \pythonil{y} to also be the new value of \pythonil{y}, and the present value of \pythonil{x} to become the new value of \pythonil{z}.
\pythonil{print(f"x={x}, y={y}, z={z}")} now gives us \textil{x=1, y=2, z=3}.
%
\FloatBarrier%
\endhsection%
%
