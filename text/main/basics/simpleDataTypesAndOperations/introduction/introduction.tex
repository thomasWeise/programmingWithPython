\hsection{Introduction}%
%
We now know how to create and run \python\ programs, both in the \pgls{IDE} and \pgls{terminal}.
We have also already learned our first two \python\ commands:%
%
\begin{itemize}%
\item \pythonil{print("Hello World!")} prints the text \inQuotes{Hello World!} to the output.
\item \pythonil{exit()} exits and terminates the \python\ interpreter.%
\end{itemize}%
%
Now, it would be very strange if the \pythonil{print} function could print \inQuotes{Hello World!}.
That would not make much sense.
\pythonil{print} expects one parameter.
This parameter cannot just be anything.
It must be a text.

The command \pythonil{exit}, on the other hand, can either have no parameter or one parameter.
If it receives one parameter, this parameter will be the exit code of the program.
Here, \pythonil{0} indicates success.
If no parameter is provided, this will be used as default value.
We need to invoke \pythonil{exit} if we use the \python\ console in the \pgls{terminal} explicitly.
If we just run a program, then after the last instruction of the program was executed, then the interpreter will also terminate with exit code~0.
Indeed, when we executed our first program in \cref{sec:ourFirstProgram}, we saw exactly that happen in \cref{fig:firstProgram09programResult}.
Different from the parameter of \pythonil{print}, which must be some text, the parameter of \pythonil{exit} needs to be a number.

We realize:
Distinguishing different types of data makes sense.
Sometimes we need to do something with text.
Sometimes we want to do something with numbers.
Somtimes, we want to just handle a decision which can be either \inQuotes{yes} or \inQuotes{no}.

Of course, for these different situations, different possible operations may be useful.
For example, when we use numbers, we may want to divide or multiply them.
When we handle text, we may want to concatenate two portions of text, or maybe we want to convert lowercase characters to uppercase.
We may want to do something if two decision variables are both \inQuotes{yes} or, maybe, if at least one of them is.

In this chapter, we will look into the simple datatypes of \python, namely:%
%
\begin{itemize}%
%
\item \pythonilIdx{int}: the integer datatype, which represents integers numbers~\integerNumbers~(\cref{sec:int}),%
\item \pythonilIdx{float}: the floating point numbers, i.e., a subset of the real numbers~\realNumbers~(\cref{sec:float}),%
\item \pythonilIdx{bool}: Boolean values, which can be either \pythonilIdx{True} or \pythonilIdx{False}~(\cref{sec:bool}),%
\item \pythonilIdx{str}: strings, i.e., portions of text of arbitrary length~(\cref{sec:str}), and%
\item \pythonilIdx{None}: nothing, which is the result of any command that does not explicitly return a value~(\cref{sec:none}).%
%
\end{itemize}%
%
\endhsection%
%
