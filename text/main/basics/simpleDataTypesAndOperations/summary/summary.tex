\hsection{Summary}%
%
This section was far more exhausting than what I initially anticipated.
I admit that.
But I think we now have a solid understanding of the simple datatypes that \python\ offers to us and what we can do with them.
We learned about integer numbers and how we can do arithmetics with them.
We learned about floating numbers, which can represent fractions and which are limited in their precision.
Boolean expressions, such as comparisons, can either be \pythonil{True} or \pythonil{False}.
Strings are used to represent text and we learned how to convert the other types to and from them.
Finally, \pythonil{None} represents the absense of any value.
Equipped with this knowledge, we now can embark to learn how to write programs that compute with these datatypes.%
%
\endhsection%
%
