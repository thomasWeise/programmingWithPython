\hsection{Getting Started}%
%
This course should be a practical course, so we should get started with practical things right away.
In order to do practical things, we need to have all the necessary software on our computer.
What software is necessary to do \python\ programming?
Well, first of all, \python.
If \python\ is not yet installed on your machine, then we briefly outline in \cref{sec:installingPython} how you can install it.

Now with the programming language \python\ alone, you cannot really do much -- in a convenient way, at least.
You need a nice editor in which you can write the programs.
Actually, you want an editor where you can not just write programs.
You want an editor  where you can also directly execute and test your programs.
In software development, you often work with a \gls{VCS} like \gls{git}.
You want to do that convenient from your editor.
Such an editor, which integrates many of the common tasks that occur during programming, is called an \gls{IDE}.
In our course, you will work with the \pycharm\ \gls{IDE}.
If you do not yet have \pycharm\ installed, in \cref{sec:installingPyCharm} we give a brief summary on how you can install it.%
%
\hinput{installingPython}{installingPython}%
\hinput{installingPyCharm}{installingPyCharm}%
%
\endhsection%
%
