\hsection{Getting Started}%
%
This course should be a practical course, so we should get started with practical things right away.
In order to do practical things, we need to have all the necessary software on our computer.
What software is necessary to do \python\ programming?
Well, first of all, \python.
If \python\ is not yet installed on your machine, then you can follow brief installation guide in \cref{sec:installingPython}.

Now with the programming language \python\ alone, you cannot really do much -- in a convenient way, at least.
You need a nice editor in which you can write the programs.
Actually, you want an editor where you can not just write programs.
You want an editor  where you can also directly execute and test your programs.
In software development, you often work with a \pgls{VCS} like \git.
You want to do that convenient from your editor.
Such an editor, which integrates many of the common tasks that occur during programming, is called an \pgls{IDE}.
In our course, you will work with the \pycharm\ \pgls{IDE}~\cite{VHN2023HOADWP}.
If you do not yet have \pycharm\ installed, then you can work through the setup instructions outlined in \cref{sec:installingPyCharm}.

Before we get into these necessary installation and setup steps that we need to really learn programming, we face a small problem:
Today, devices with many different \pgls{OS} are available.
For each \pgls{OS}, the installation steps and software availability may be different, so I cannot possibly cover them all.
Personally, I strongly recommend using \linux~\cite{T1999TLE} for programming, work, and research.
If you are a student of computer science or any related field, then it is my personal opinion that you should get familiar with this operating system.
Maybe you could start with the very easy-to-use \ubuntu\ \linux~\cite{CN2020ULB}.
Either way, in the following, I will try to provide examples and instructions for both \ubuntu\ and the commercial Microsoft \windows~\cite{B2023W1IO} \pgls{OS}.
%
\hinput{installingPython}{installingPython}%
\hinput{installingPyCharm}{installingPyCharm}%
%
\endhsection%
%
