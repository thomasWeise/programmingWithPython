\hsection{Installing PyCharm}%
\label{sec:installingPyCharm}%
%
\begin{figure}%
\centering%
\includegraphics[width=0.5\linewidth]{\currentDir/usingNotepadAsEditor.pdf}%
\caption{We do not want to use the \textil{NotePad}\nobreakdashes-App of \microsoftWindows\ for programming, do we?}%
\label{fig:usingNotepadAsEditor}%
\end{figure}%
%
\begin{figure}%
\centering%
\tightbox{\includegraphics[width=0.5\linewidth]{\currentDir/unifiedPyCharm}}%
\caption{%
Some time after we finished preparing the installation instructions for the \pycharm\ Community Edition that you can find in the remainder of this section, it was announced that a unified \pycharm\ version will replace the Community Edition. %
Therefore, the download instructions will probably change. %
We will eventually update the instructions, but not now.%
}%
\label{fig:unifiedPyCharm}%
\end{figure}%
%
Just having a programming language and the corresponding interpreter on your system is not enough.
Well, it is enough for just running \python\ programs.
But it is not enough if you want to develop software efficiently.
Are you going to write programs in a simple text editor like a caveperson?
No, of course not, you need an \pgls{ide}, a program which allows you to do multiple of the necessary tasks involved in the software development process under one convenient user interface.
For this book, I recommend using \pycharm~\cite{VHN2023HOADWP,Y2022PPADT,W2024PME}, whose Community Edition is/\emph{was} freely available.
The installation guide for \pycharm\ can be found at \url{https://www.jetbrains.com/help/pycharm/installation-guide.html}.

Notice that, as shown in \cref{fig:unifiedPyCharm}, the \pycharm\ Community Edition will be/has been replaced with a unified edition.
This means that the instructions in the following are probably outdated, but they should still give you a reasonably good impression on what needs to be done.
We will probably eventually replace them~{\dots}~but not now.%
%
\FloatBarrier%
\hinput{installingPyCharmUbuntu}{installingPyCharmUbuntu}%
\hinput{installingPyCharmWindows}{installingPyCharmWindows}%
\FloatBarrier%
\endhsection%
%
