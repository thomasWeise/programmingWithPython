\hsection{Summary}%
%
If you have managed to fight your way through the book until this point, then you can already do quite a few things.
You can use the computer like a fancy calculator by evaluating numerical expressions, which you learned in
\cref{sec:simplyDataTypesAndOperations}.
By utilizing variables as discussed in \cref{sec:variables}, you can realize some simple algorithms like approximating~\numberPi.
In this section, you learned about compound datastructures that can store multiple values.
You learned about mutable and immutable sequences of values, namely lists and tuples.
You also learned that \python\ offers you the mathematical notion of sets as well as dictionaries, which are key-value mappings.
You now have the basic knowledge of the most commonly used datatypes in \python.
You know operations to manipulate and query them.

However, when learning about these structures, you may have felt some eerie awkwardness.
Something did not feel right.
If I can only construct a list step by step, then what is the advantage compared to just using many variables?
The missing ingredient are control flow statements.
Statements that allow us to iterate over a list.
Statements that allow us to perform an action~$A$ if an element is contained in a set, for example, and an action~$B$ otherwise.
The next big part of this book will close this gap and provide you with all the tools necessary to write \inQuotes{real} programs.
Later, in \cref{sec:enumOverSequences,sec:iteration}, we will circle back to the topic of collections and look into how sequences of elements can be processed iteratively.%
%
\endhsection%
%
