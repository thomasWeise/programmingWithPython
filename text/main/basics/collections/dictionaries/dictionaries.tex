\hsection{Dictionaries}%
\label{sec:dictionaries}%
%
\gitPythonAndOutput{\programmingWithPythonCodeRepo}{02_collections}{dicts_1.py}{--args format}{dicts:dicts_1}{%
An example of using dictionaries in \python.}%
%
Dictionaries in \python\ are containers that store key-value pairs.
The value associated with a given key is accessed via the \pythonil{[...]}\pythonIdx{[\idxdots]}-based indexing.
The keys are unique and can be of an arbitrary type, as long as they are immutable.
The concept of dictionary is also known as hash table or hash map in other languages or domains.
Sets and dictionaries in \python\ are implemented using similar datastructures~\cite{B2023T}, namely hash tables~\cite{K1998SAS,CLRS2009ITA,SKS2019DSC}, and thus exhibit similar performance.%
%
\begin{sloppypar}%
\cref{lst:dicts:dicts_1} shows some examples of how we can use dictionaries.
A dictionary can be created again using the \pythonil{\{...\}}\pythonIdx{\textbraceleft\idxdots\textbraceright!dict} syntax, however, this time, we place comma-separated \pythonil{key-value} pairs within the curly braces.
The \pglspl{typeHint} for dictionaries\pythonIdx{dict!type hint} is \pythonil{dict[keyType][valueType]}, where \pythonil{keyType} is the type for the keys and \pythonil{valueType} is the type for the values~\cite{PEP585}.
The line \pythonil{num_str: dict[int, str] = \{2: "two", 1: "one", 3: "three", 4: "four"\}} therefore creates a new dictionary and stores it in the variable \pythonil{num_str}.
The type-hint of the variable states that it can point only to dictionaries that have integers as keys and strings as values.
The contents of the dictionary are the key-value pairs \pythonil{2: "two"}, \pythonil{1: "one"}, \pythonil{3: "three"}, and \pythonil{4: "four"}.
This means that the value string \pythonil{"one"} is stored under the key \pythonil{1}, the value string \pythonil{"two"} is stored under the key \pythonil{2}, and so on.%
\end{sloppypar}%
%
We can use dictionaries in \pglspl{fstring} like any other \python\ object.
The function \pythonilIdx{len}\pythonIdx{dict!len} can give us the number of elements in a dictionary.
For our dictionary \pythonil{num_str} with four key-value pairs, it returns \pythonil{4}.
The elements of the dictionary can be accessed by their keys using square brackets.
Therefore \pythonil{num_str[2]} will return the string~\pythonil{"two"} associated with key~\pythonil{2} and \pythonil{num_str[1]} returns the string~\pythonil{"one"}.%
%
\begin{sloppypar}%
We can access the keys, values, and key-value pairs of a dictionary as collections using the \pythonilIdx{keys}\pythonIdx{dict!keys}, \pythonilIdx{values}\pythonIdx{dict!values}, and \pythonilIdx{items}\pythonIdx{dict!items} functions, respectively.
In the example \cref{lst:dicts:dicts_1}, we convert them to lists:
\pythonil{list(num_str.keys())}\pythonIdx{keys}\pythonIdx{dict.keys} gives us all the keys in the dictionary \pythonil{num_str} as a \pythonilIdx{list}, which thus equals \pythonil{[2, 1, 3, 4]}.
Similarly, \pythonil{list(num_str.values())}\pythonIdx{values}\pythonIdx{dict.values} gives us all the values in the dictionary \pythonil{num_str} as a \pythonilIdx{list}, which thus equals \pythonil{["two", "one", "three", "four"]}.
Finally, \pythonil{num_str.items()}\pythonIdx{items}\pythonIdx{dict.items} returns the key-value pairs a sequence of tuples and wrapping this into a list yields \pythonil{[(2, "two"), (1, "one"), (3, "three"), (4, "four")]}.
You may notice that all of these sequences have the same order of elements that was used when we created the dictionary.
They key-value pair \pythonil{2: "two"} comes before \pythonil{1: "one"}.
This is because dictionaries, different from sets, are ordered datastructures.
Their elements appear in all sequenced versions always in the same order in which they were originally inserted into the dictionary~\cite{PSF2024D}.%
\end{sloppypar}%
%
\begin{sloppypar}%
Dictionaries can be modified.
We can add an entry simply by assigning a value to a key.
\pythonil{num_str[5] = "fivv"} assigns the value \pythonil{"fivv"} to the key~\pythonil{5}.
This association is now part of the dictionary.
Oops, we noticed a typo:
We wanted to add \pythonil{"five"}, not \pythonil{"fivv"}.
\pythonil{num_str[5] = "five"} will overwrite this faulty assignment.
All keys in a dictionary are unique, so there can only be one value associated with key~\pythonil{5}.
This also means that dictionary keys must obeye the same requirement as set elements:%
\end{sloppypar}%
%
\bestPractice{dictKeys}{Dictionary keys must be immutable.}%
%
We can also delete key-value pairs.
This is done by invoking \pythonil{del dict[key]}\pythonIdx{del}\pythonIdx{dict!del}, which deletes the key~\pythonil{key} and its associated value from the dictionary.
The method \pythonilIdx{pop}\pythonIdx{dict!pop} allows us to get the value of a given key and then immediately deletes the key-value pair from the dictionary.
\pythonil{num_str.pop(5)} will return \pythonil{"five"} and remove the key-value pair~\pythonil{5: "five"} from the dictionary.
Afterwards, requesting \pythonil{num_str[5]} would lead to an \pythonIdx{KeyError} which would terminate the program (unless properly caught).%
%
\begin{sloppypar}%
As stated before, the types for keys and values in a dictionary can be chosen arbitrary, with the limitation that keys must be immutable.
We now create the empty dictionary \pythonil{str_num: dict[str, int] = \{\}}\pythonIdx{\textbraceleft\textbraceright}\pythonIdx{dict!empty}.\footnote{%
Using the \pythonil{dict()}\pythonIdx{dict!empty} without arguments has the same effect, but several \pglspl{linter} discourage this and encourage using \pythonil{\{\}}\pythonIdx{\textbraceleft\textbraceright} instead.%
}
Notice that only due to the \pgls{typeHint} \pythonil{dict[str, int]}, other programmers and static type-checking tools can know at this point that we intend to use strings as keys and integers as values in this new dictionary.
The \pythonilIdx{update}\pythonIdx{dict!update} method allows us to append the values of an existing dictionary to a given dictionary.
\pythonil{str_num.update(\{"one": 1, "three": 3, "two": 2, "four": 4\})} therefore stores four key-value pairs into our new dictionary.
This time, the keys are indeed strings and the values are integers.
The slightly convoluted \pgls{fstring} \pythonil{f"\{num_str[1]\} + \{num_str[2]\} = \{str_num['three']\}"} evaluates to \pythonil{"one + two = 3"}.
The first two values are looked up in the \pythonil{num_str} dictionary under keys \pythonil{1} and \pythonil{2}, respectively.
The last value is obtained from \pythonil{str_num} under key~\pythonil{"three"}.%
\end{sloppypar}%
%
We will discuss a bit more about the mechanism that allows us to store and retrieve objects from dictionaries in~\cref{sec:hashDunder}.%
%
\FloatBarrier%
\endhsection%
%
