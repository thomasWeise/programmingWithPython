\hsection{Tuples}%
\label{sec:tuples}%
%
\gitPythonAndOutput{\programmingWithPythonCodeRepo}{02_collections}{tuples_1.py}{--args format}{tuples:tuples_1}{%
A first example for using tuples in \python: creating, indexing, and printing of tuples.}%
%
\gitPythonAndOutput{\programmingWithPythonCodeRepo}{02_collections}{tuples_2.py}{--args format}{tuples:tuples_2}{%
A second example for using tuples in \python: tuples with elements of different types and tuple unpacking.}%
%
\gitPythonAndErrorOutput{\programmingWithPythonCodeRepo}{02_collections}{tuples_3.py}{--args format}{tuples:tuples_3}{%
A third example for using tuples in \python: testing the immutability property.}%
%
\gitOutputTool{\programmingWithPythonCodeRepo}{.}{scripts/mypy.sh 02_collections tuples_3.py}{tuples:tuples_3:mypy}{%
The results of static type checking with \mypy\ of the program given in \cref{lst:tuples:tuples_3}.}%
%
\pythonIdx{tuple}%
Tuples are very similar to lists, with three differences:
First, they are immutable.
You cannot add, delete, or change the elements of a tuple.
Second, on a semantic level, lists are intended to hold objects of the same type.
The \pgls{typeHint}~\pythonil{list[int]} indicates that we want something to be list of integer numbers.
While the \python\ interpreter permits us to ignore this and still store arbitrary objects in that list, this would violate the idea behind lists.
Tuples, on the other hand, are designed to contain elements of different types.
Since they cannot be changed, it will always be clear which element of which type is at which location.
Three, tuples are defined using parentheses instead of square brackets, i.e., with~\pythonil{(...)}\pythonIdx{(\idxdots)}.%
%
\bestPractice{listOrTuple}{%
When you need to use an indexable sequence of objects, use a list if you intent to modify this sequence. %
If you do not intent to change the sequence, use a tuple.}%
%
\begin{sloppypar}%
\Cref{lst:tuples:tuples_1} shows us some of the things we can do with tuples.
We create a tuple \pythonil{fruits} to hold three strings.
The proper \pgls{typeHint} for this is \pythonil{tuple[str, str, str]}\pythonIdx{tuple!type hint}~\cite{PEP585}.
The line \pythonil{fruits: tuple[str, str, str] = ("apple", "pear", "orange")} thus creates a tuple~\pythonil{fruits} which contains three strings, namely \pythonil{"apple"}, \pythonil{"pear"}, and~\pythonil{"orange"}
We also annotated the variable with a \pgls{typeHint} informing any static code analysis tool that we indeed intend to store three strings in the tuple.
Notice that the tuple is defined using parentheses, whereas a list with the same content would have been defined using square brackets, e.g., as \pythonil{fruits: list[str] = ["apple", "pear", "orange"]}.%
\end{sloppypar}%
%
\begin{sloppypar}%
If we are not sure about the actual number of elements that we will put in a tuple but know that they are all of the same type, we can use the ellipsis~\pythonilIdx{...} in the \pgls{typeHint}.
\pythonil{tuple[str, ...]} denotes a tuple that can receive an arbitrary amount of strings.\pythonIdx{tuple!type hint!...}
The line \pythonil{veggies: tuple[str, ...] = ("onion", "potato", "leek", "garlic")} is therefore correct and defined a tuple~\pythonil{veggies} consisting of four strings.%
\end{sloppypar}%
%
The elements of tuples can be accessed using the normal square brackets, exactly like list elements.
\pythonil{veggies[0]} gives us the first element of the tuple~\pythonil{veggies}, namely \pythonil{"onion"}.
\pythonil{veggies[1]} gives us the second element of the tuple~\pythonil{veggies}, namely \pythonil{"potato"}.
Negative indices also work and index the tuple from the end.
\pythonil{veggies[-1]} gives us the last element of the tuple~\pythonil{veggies}, namely \pythonil{"garlic"}.
\pythonil{veggies[-2]} gives us the second-to-last element of the tuple~\pythonil{veggies}, namely \pythonil{"leek"}.

Tuples support many of the same non-modifying operators as lists.
The \pythonilIdx{in}\pythonIdx{tuple!in} and \pythonilIdx{not in}\pythonIdx{tuple!not in} operators both work.
We quickly test only the former by asking \pythonil{'pear' in fruits}, which evaluates to~\pythonil{True} and \pythonil{'pear' in veggies}, which is~\pythonil{False}.
The \pythonilIdx{index}\pythonIdx{tuple!index}, too, is available and \pythonil{fruits.index('apple')} returns~\pythonil{0} because \pythonil{"apple"} is the very first element of~\pythonil{fruits}.
%
\begin{sloppypar}%
In \cref{lst:tuples:tuples_2} we explore tuples containing elements of multiple types.
The \pgls{typeHint} \pythonil{tuple[str, int, float]} states that we want to define a tuple where the first element is a string, the second element is an integer number, and the third element is a floating point number.
The line \pythonil{mixed: tuple[str, int, float] = ("apple", 12, 1e25)} stores such a tuple in the variable~\pythonil{mixed}.
The first element of the tuple is the string~\pythonil{"apple"}.
The second element is the integer~\pythonil{12} and the third element is the floating point number~\pythonil{1e25}, i.e.,~$10^{25}$.
With the line \pythonil{other: tuple[str, int, float] = ("pear", 1, 1.2)} we create another such tuple.%
\end{sloppypar}%
%
\begin{sloppypar}%
We now want to create a list of such tuples.
This list should, of course, also be annotated with the proper \pgls{typeHint}.
This is easy:
The \pgls{typeHint} for our kind of tuples is \pythonil{tuple[str, int, float]}.
The \pgls{typeHint} for a list is \pythonil{list[elementType]}, where \pythonil{elementType} is the type of the elements.
If we want to create a list containing tuples of our kind, then the proper \pgls{typeHint} would be \pythonil{list[tuple[str, int, float]]}.
Having realized this, we can now create the list~\pythonil{tuples}.
We use the square bracket syntax for this.
As first element, we put the tuple stored in the variable~\pythonil{mixed}.
Then we simply define another tuple~\pythonil{("pear", -2, 4.5)} inline.
The third element is the tuple stored in variable~\pythonil{other}.
As last element, we again define a tuple~\pythonil{("pear", -2, 3.3)}.
\Cref{exec:tuples:tuples_2} shows that we can print this list of tuples just like any other list (using an~\pgls{fstring}).%
\end{sloppypar}%
%
There is an order defined on tuples.
Two tuples~\pythonil{x} and~\pythonil{y} are compared elementwise lexicographically.
If the first element of~\pythonil{x} is less than the first element of~\pythonil{y}, then~\pythonil{x < y}.
If the first element of~\pythonil{x} is greater than the first element of~\pythonil{y}, then~\pythonil{x > y}.
If the first elements of both tuples are the same, the comparison continues at the second element, and so on.
This means that we can sort our list of tuples by writing~\pythonil{tuples.sort()}\pythonIdx{list!sort}.
As a result, the tuple with the string \pythonil{"apples"} remains at the first position.
All other tuples have \pythonil{"pear"} as first element and thus are \inQuotes{greater.}
Two of them have the integer~\pythonil{-2} as the second element and thus are located before the tuple with~\pythonil{1} as second element.
Among these two tuples, the one with the smallest third element comes first.

Finally, this example program shows us that we can unpack tuples just like lists\pythonIdx{tuple!unpacking}\pythonIdx{unpacking}.
\pythonil{a, b, c = mixed} stores \pythonil{"apple"} in~\pythonil{a}, \pythonil{12} in~\pythonil{b}, and~\pythonil{1e25} in~\pythonil{c}.
Interestingly, we can also create tuples the other way around.
We can leave away the parentheses when creating tuples in cases where no confusion is possible.
\pythonil{mixed = "x", 4, 4.5} has the same effect as \pythonil{mixed = ("x", 4, 4.5)}.

Let us finally explore the immutability of tuples in \cref{lst:tuples:tuples_2}.
As stated before, we are not permitted to add, remove, insert, delete, or overwrite any element of a tuple.
What happens if we try create a tuple that contains a list?
The line \pythonil{mt: tuple[int, list[int]] = (1, [2])} does just this.
It creates a new tuple~\pythonil{mt} whose first element is the integer number~\pythonil{1} and whose second element is a list of integers, namely~\pythonil{[2]}.
We can access this list via~\pythonil{mt[1]}.
While the tuple cannot be changed, this list can!
\pythonil{mt[1].append(2)} will append the number~\pythonil{2} to the list, which now is~\pythonil{[2, 2]}.
Printing \pythonil{f"mt == {mt}"} therefore yields~\textil{mt == (1, [2, 2])}.

What we are never permitted to do is to change the contents of the tuple itself.
Trying to do \pythonil{mt[1] = [3, 4]} will raise a~\pythonilIdx{TypeError}.
This is an exception which, unless properly caught and handled, will terminate the program immediately.
\Cref{exec:tuples:tuples_3} shows us this.
While the first two \pythonilIdx{print} commands still succeed, the program crashes when we lay our hands on the tuple directly.
The \python\ interpreter is kind enough to print the error as well as where it happened.
This would help the programmer to find the problem.

Had the programmer used \mypy\ to check the code before running it, however, they would have found the error already.
The output of \mypy\ given in \cref{exec:tuples:tuples_3:mypy} shows us that a tuple is an \emph{Unsupported target for indexed assignment}.
And indeed it is.%
%
\bestPractice{noModifiableObjectInATuple}{%
Only put immutable objects into tuples. %
Mutable objects inside tuples makes the tuples modifiable as well, while other programmers may assume that they are immutable. %
This can lead to strange errors down the line.%
}%
%
In summary, tuples are like lists, just immutable and they are semantically conceptualized to contain objects of different types.
Careful with immutability, though, as it can only be guaranteed if the tuple itself consists of only immutable objects.%
%
\FloatBarrier%
\endhsection%
%
