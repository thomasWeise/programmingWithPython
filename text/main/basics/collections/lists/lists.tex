\hsection{Lists}%
\label{sec:lists}%
%
\gitLoadAndExecPython{lists:lists_1}{}{collections}{lists_1.py}{}%
\listingPythonAndOutput{lists:lists_1}{%
A first example for using lists in \python:~creating, indexing, printing of and appending elements and other lists to lists.}{}%
%
\gitLoadAndExecPython{lists:lists_2}{}{collections}{lists_2.py}{}%
\listingPythonAndOutput{lists:lists_2}{%
A second example of using lists in \python: inserting and deleting elements, sorting and reversing lists.}{}%
%
\gitLoadAndExecPython{lists:lists_3}{}{collections}{lists_3.py}{}%
\listingPythonAndOutput{lists:lists_3}{%
A third example of using lists in \python: slicing, adding, and multiplying lists.}{}%
%
A \pythonilIdx{list} is a mutable sequence of objects which can be accessed via their index~\cite{PSF:P3D:TPLR:S}.
They work very similar to the strings we already discussed in \cref{sec:str}, but instead of (only) characters, they can contain any kind of objects and they can be modified.

With program \programUrl{lists:lists_1} in \cref{lst:lists:lists_1}, we provide some first examples for using lists.
A list can be defined by simply writing its contents, separated by~\pythonilIdx{,} inside square brackets~\pythonil{[...]}\pythonIdx{[\idxdots]}.
\pythonil{["apple", "pear", "orange"]} creates a list with three elements, namely the strings \pythonil{"apple"}, \pythonil{"pear"}, and~\pythonil{"orange"}.
If we want to store a list in a variable, then we can use the \pgls{typeHint}~\pythonil{list[elementType]}\pythonIdx{list!type hint} where \pythonil{elementType} is to be replaced with the type of the list elements~\cite{PEP585}.
\pythonil{fruits: list[str] = ["apple", "pear", "orange"]} therefore creates the list \pythonil{fruits} with the contents listed above.
It also tells any automated type checking tool and other programmers that we intent that only \pythonil{str} values should be stored inside the list.

The length of a list is can be ob\pythonIdx{list!unpacking}\pythonIdx{unpacking}tained using the \pythonilIdx{len}\pythonIdx{list!len} function.
\pythonil{len(fruits)} will therefore return the value~\pythonil{3}.
We can use lists in \pglspl{fstring} just like any other datatype.
The string representation of \pythonil{fruits} which then would be used is simply \pythonil{"['apple', 'pear', 'orange']"}.%
%
\begin{sloppypar}%
We can add single elements to a list by using the \pythonilIdx{append}\pythonIdx{list!append} method.
Invoking \pythonil{fruits.append("cherry")} will append the string \pythonil{"cherry"} to the list \pythonil{fruits}.
The list then equals \pythonil{["apple", "pear", "orange", "cherry"]} and has \pythonil{len(fruits) == 4}.%
\end{sloppypar}%
%
Of course we can have multiple lists in a program.
In \cref{lst:lists:lists_1}, we now create the second list \pythonil{vegetables} with the three elements \pythonil{"onion"}, \pythonil{"potato"}, and~\pythonil{"leek"}.

An empty list is created with expression~\pythonilIdx{[]}\pythonIdx{empty}, which consists of just the square brackets with no contents inside.
We can append \emph{all} the elements of one collection to a list by using the \pythonilIdx{extend}\pythonIdx{list!extend} method.
We start with the empty list~\pythonil{food} and then invoke~\pythonil{food.extend(fruits)}.
Now all the contents of the list \pythonil{fruits} are appended to \pythonil{food}.
We then invoke~\pythonil{food.extend(vegetables)}, which will add all the elements from the list \pythonil{vegetables} to \pythonil{food} as well.
\pythonil{fruits} and \pythonil{vegetables} remain unchanged during this procedure, but \pythonil{food} now contains all of their elements as well.
It contains all seven fruits and vegetables and its \pythonil{len(food)} is therefore~\pythonil{7}.

We can access the elements of a list by their index, again in the same way we access the characters in a string.
\pythonil{food[0]} returns the first element of the list \pythonil{food}, which is \pythonil{"apple"}.
\pythonil{food[1]} returns the second element of the list \pythonil{food}, which is \pythonil{"pear"}.
And so on.
We can also access the elements using the end of the list as reference:
\pythonil{food[-1]} returns the last element of the list \pythonil{food}, which is \pythonil{"leek"}.
\pythonil{food[-2]} returns the second-to-last element of the list \pythonil{food}, which is \pythonil{"potato"}.
And so on.

Finally, elements can also be deleted from the list by their index.
\pythonIdx{del}\pythonIdx{list!del}\pythonil{del food[1]} deletes the second element from the list~\pythonil{food}.
The second element is \pythonil{pear} and if we print \pythonil{food} again, it has indeed disappeared.

In program \programUrl{lists:lists_2} in \cref{lst:lists:lists_2}, we illustrate some more operations on lists.
We begin again by creating a list, this time of numbers: \pythonil{numbers: list[int] = [1, 7, 56, 2, 4]} creates (and type-hints) a list of five integers.
If we want to know whether an element is included in a list, we can use the \pythonilIdx{in}~operator\pythonIdx{list!in}.
\pythonil{7 in numbers} returns \pythonil{True} if \pythonil{7} is located somewhere inside the list~\pythonil{numbers} and \pythonil{False} otherwise.
The \pythonilIdx{not in}\pythonIdx{list!not in} operator inquires the exact opposite:
\pythonil{2 not in numbers} becomes \pythonil{True} if \pythonil{2} is \emph{not} in the list~\pythonil{numbers} and is \pythonil{False} if it is in the list.
If we want to know at which index a certain element in the list is located, we can use the \pythonilIdx{index}\pythonIdx{list!index} method.
\pythonil{numbers.index(7)} will search where the number~\pythonil{7} is located inside~\pythonil{numbers}.
Since it is the second elements and indices start at~0, it returns~\pythonil{1}.
Similarly, \pythonil{numbers.index(7)} returns~\pythonil{3}, because \pythonil{numbers[3] == 2}.

The \pythonilIdx{insert}\pythonIdx{list!insert} method allows us insert an element at a specific index.
The elements which are currently at that or higher indices are moved up one slot.
\pythonil{numbers.insert(2, 12)} will insert the number~\pythonil{12} at index~\pythonil{2} into the list~\pythonil{numbers}.
The element~\pythonil{56} which currently occupies this spot is moved to index~\pythonil{3}, which means that the~\pythonil{2} located at this place is moved to index~\pythonil{4}, which means that the value~\pythonil{4} which right now is stored in this location will move to index~\pythonil{5}.
The list \pythonil{numbers} now looks like this: \pythonil{[1, 7, 12, 56, 2, 4]}.

If we want to remove a specific element from the list without knowing its location, the \pythonilIdx{remove}\pythonIdx{list!remove} method will do the trick.
\pythonil{numbers.remove(56)} searches through the list~\pythonil{numbers} for the element~\pythonil{56} and, once it finds it, deletes it.
The list becomes~\pythonil{[1, 7, 12, 2, 4]}.

We can sort a list inplace by using the~\pythonilIdx{sort}\pythonIdx{list!sort} method.
\pythonil{numbers.sort()} sorts the list~\pythonil{numbers}, which then becomes~\pythonil{[1, 2, 4, 7, 12]}.
Similarly, we can reverse a list, i.e., make the last element become the first, the second-to-last element the second, and so on, by using the method~\pythonil{reverse}\pythonIdx{list!reverse}.
Reversing the list~\pythonil{numbers} after we sorted it will turn it into~\pythonil{[12, 7, 4, 2, 1]}.

If we want to create a list copy of an existing sequence, we can just invoke the constructor~\pythonilIdx{list} directly.
\pythonil{cpy: list[int] = list(numbers)} creates the new list~\pythonil{cpy} which has the same contents as~\pythonil{numbers}.
This means that \pythonil{cpy == numbers} will be~\pythonil{True}, because \pythonil{cpy} is an exact copy of~\pythonil{numbers}.
\pythonil{cpy is numbers}, however, is~\pythonil{False}.
They are not the same object.

We can change~\pythonil{cpy} by deleting its first element via~\pythonil{del cpy[0]}.
\pythonil{numbers} will be unaffected by this and stays unchanged.
Now, both \pythonil{cpy == numbers}\pythonIdx{==} and \pythonil{cpy is numbers}\pythonIdx{is} will be~\pythonil{False}.
By the way, in the same manner as \pythonilIdx{not in} is the opposite of the \pythonil{in}~operator, \pythonilIdx{not is} is the opposite of~\pythonilIdx{is}:
\pythonil{cpy is not numbers} yields \pythonil{True} because \pythonil{cpy} is not the same object as \pythonil{numbers}.

Program \programUrl{lists:lists_3} given in \cref{lst:lists:lists_3} continues our journey through the magical land of \python\ \pythonils{list}.
You can add two lists~\pythonil{a} and~\pythonil{b} using \pythonilIdx{+}\pythonIdx{list!+}, i.e., do \pythonil{c = a + b}.
The result is a new list which contains all elements of the first list~\pythonil{a} followed by all the elements from the second list~\pythonil{b}.
Therefore, the expression \pythonil{[1, 2, 3, 4] + [5, 6, 7]} results in \pythonil{[1, 2, 3, 4, 5, 6, 7]}.
Similarly, if you multiply a list by an integer~\pythonil{n}, you get a new list which equals the old list \pythonil{n}~times concatenated.
\pythonil{[5, 6, 7] * 3}\pythonIdx{*!list}\pythonIdx{list!*} therefore yields \pythonil{[5, 6, 7, 5, 6, 7, 5, 6, 7]}.

In \cref{sec:strBasicOperations}, we discussed string slicing.
Lists can be sliced in pretty much the same way\pythonIdx{list!slicing}\pythonIdx{slicing}\pythonIdx{slice}~\cite{PSF:P3D:TPLR:S}.
When slicing a list~\pythonil{l} or a string, you can provide either two or three values in the square brackets\pythonIdx{[\idxdots]}\pythonIdx{[i:j:k]}, i.e., either do~\pythonil{l[i:j]}\pythonIdx{[i:j]}\pythonIdx{slicing} or \pythonil{l[i:j:k]}.
If \pythonil{j < 0}, then it is replaced with~\pythonil{len(l) - j}\pythonIdx{len}.
In both the two and three indices case, \pythonil{i} is the inclusive start index and \pythonil{j} is the exclusive end index, i.e., all elements with index~\pythonil{m} such that \pythonil{i <= m < j}.
In other words, the slice will contain elements from~\pythonil{l} whose index is between~\pythonil{i} and~\pythonil{j}, including the element at index~\pythonil{i} but \emph{not} including the element at index~\pythonil{j}.
If a third index~\pythonil{k} is provided, the it is the step length.
\pythonil{l[i:j:k]} selects all items at indices~\pythonil{m} where \pythonil{m = i + n*k}, \pythonil{n >= 0} and \pythonil{i <= m < j}.
If~\pythonil{i} is omitted, i.e., if \pythonil{:j:k} is provided, then \pythonil{i=0} is assumed.
If~\pythonil{j} is omitted, i.e., if \pythonil{i::k} is provided, then \pythonil{j=len(l)} is assumed.

If you have a list \pythonil{lst4 = [5, 6, 7, 5, 6, 7, 5, 6, 7]}, then the slice~\pythonil{lst4[2:-2]} will return a new list which contains all the elements of~\pythonil{lst4} starting from index~\pythonil{2} and up to (excluding) the second-to-last element.
The slice~\pythonil{lst4[1::2]} starts at index~\pythonil{1}, continues until the end of the list, and adds every second element.
This results in~\pythonil{[6, 5, 7, 6]}.
As final example, consider the slice~\pythonil{lst4[-1:3:-2]}.
It will begin creating the new list at the last element.
The step-length is~\pythonil{-2}, so it will move backwards and add every second element to the new list.
It stops adding elements before reaching index~\pythonil{3}.
Therefore, the result will be the new list~\pythonil{lst7 = [7, 5, 6]}.

Notice that the slices\pythonIdx{slice} we create are independent copies of ranges of the original lists.
The list~\pythonil{lst7} is a slice from the list~\pythonil{lst4}.
If we modify it, e.g., set \pythonil{lst7[1] = 12}, then we set the second element of~\pythonil{lst7} to~\pythonil{12}.
\pythonil{lst7}~becomes~\pythonil{[7, 12, 6]}.
Now, the second element of~\pythonil{lst7} originally is the seventh element of~\pythonil{lst4}, namely the \pythonil{5} located at index~\pythonil{6}, which is equivalent to index~\pythonil{-3}.
You may wonder whether this element now also has changed.
It did not.
\pythonil{lst4} remains unchanged by any operation on the independent copied slice~\pythonil{lst7}.

An interesting functionality is also list unpacking\pythonIdx{list!unpacking}\pythonIdx{unpacking}.
In \cref{sec:strBasicOperations}, the list~\pythonil{lst2} contains the three elements~\pythonil{[5, 6, 7]}.
If we know the number of elements in the list in our program, then we can assign them to exactly the same number of variables.
\pythonil{a, b, c = lst2} creates and assigns values to three variables \pythonil{a=5}, \pythonil{b=6}, and \pythonil{c=7} by unpacking the list~\pythonil{lst2}.
Of course, this only works if list~\pythonil{lst2} has exactly length~3.%
%
\FloatBarrier%
\endhsection%
%
