\hsection{Dunder Methods}%
%
In \python, \emph{everything is an object}~\cite{PSF2024OVAT,J2022PPEIAOSCD}.
Functions, modules, classes, datatypes, values of simple datatypes, and so on -- all are objects.
Many of these objects have special functionality.
For example, we can add, multiply, and divide numerical objects.
We can get string representations for all objects that we can print to the console.
We can iterate over the elements of objects that represent sequences.
We can execute objects that represent functions.
These special functionalities are implemented by so-called \emph{dunder} methods.
\pythonIdx{dunder}Dunder methods have names that begin and end with two underscores, like~\pythonilIdx{\_\_init\_\_}\pythonIdx{dunder!\_\_init\_\_}.
And indeed, \pythonilIdx{\_\_init\_\_}\pythonIdx{dunder!\_\_init\_\_}~is a dunder method, the initializer that creates the attributes of an object.

We already learned that, if we create a subclass of a class, we can define new methods and override existing ones.
We can do the same with dunder methods.
This means that we can implement, create, change, and customize all of the functionalities listed above!%
%
\hsection{\texttt{\_\_str\_\_}, \texttt{\_\_repr\_\_}, and \texttt{\_\_eq\_\_}}%
%
\gitPythonAndOutput{\programmingWithPythonCodeRepo}{09_dunder}{str_vs_repr.py}{--args format}{dunder:str_vs_repr}{%
Comparing the \pythonilIdx{str} and \pythonilIdx{repr} representations of integers, strings, lists, and \python's \pythonilIdx{datetime}\pythonIdx{datetime!datetime}\pythonIdx{datetime!datetime!now}\pythonIdx{datetime!UTC} class.}%
%
In \python, we can distinguish two forms of string representations of a given object~\pythonil{o}:%
%
\begin{itemize}%
%
\item \pythonil{str(o)}\pythonIdx{str} should return a concise and brief representation of the object~\pythonil{o}.
This representation is mainly for end users.
\pythonil{str(o)} invokes the \pythonilIdx{\_\_str\_\_}\pythonIdx{dunder!\_\_str\_\_} dunder method, if it has been implemented.
Otherwise, \pythonilIdx{\_\_repr\_\_}\pythonIdx{dunder!\_\_repr\_\_} is used instead.%
%
\item \pythonil{repr(o)}\pythonIdx{repr} should ideally return a string representation that contains all the information that is needed to re-create the object.
The target audience here are programmers who are working on the code, who may need to write precise information into log files, or who are searching for errors.
\pythonil{repr(o)} invokes the \pythonilIdx{\_\_repr\_\_}\pythonIdx{dunder!\_\_repr\_\_} dunder method, if it has been implemented.
Otherwise, it returns the default representation, which is the type name and ID of the object.%
%
\end{itemize}%
%
These two functions are compared in \cref{lst:dunder:str_vs_repr}.
Here, we first create an integer variable \pythonil{the_int} with value \pythonil{123}.
Both \pythonil{str(the_int)} and \pythonil{repr(the_int)} are \pythonil{"123"}.
This is to be expected, since this is all the information that is needed to completely recreate this value and, at the same time, it is also the most concise way to present the value.

We then create another variable \pythonil{the_str} with value \pythonil{"123"}.
Printing \pythonil{the_str} to the \pgls{stdout}, which is equivalent to \pythonil{print(str(the_str))}, will make the text \textil{123} appear on the console.
Printing \pythonil{repr(the_str)}, however, produces~\textil{'123'}.
Notice the added single quotation marks on each side?
These are necessary.
Without them, \pythonil{repr(the_str)} and \pythonil{repr(the_int)} would be the same.
We could not distinguish whether the value we printed was a string or an integer.
This, of course, matters only if we care about the internal workings of our program.
This is the purpose for the existence of~\pythonilIdx{repr}.%
%
\begin{sloppypar}%
Next, we create two collections.
First comes the list~\pythonil{l1}, which contains the three integers~\pythonil{1}, \pythonil{2}, and~\pythonil{3}.
Then we create the list~\pythonil{l2}, which contains the three strings~\pythonil{"1"}, \pythonil{"2"}, and~\pythonil{"3"}.
Then we print both lists, which will use \pythonil{str(l1)} and \pythonil{str(l2)} internally.
The result of \pythonil{print(f"\{l1 = \}, but \{l2 = \}")} is \textil{l1 = [1, 2, 3], but l2 = ['1', '2', '3']}.
Notice that the single quotation marks around the string elements of \pythonil{l2} are printed?
When obtaining the string representations of the standard \python\ collections with either \pythonilIdx{str} or \pythonilIdx{repr}, the elements of the collections are converted to strings using \pythonilIdx{repr}, not~\pythonilIdx{str}~\cite{PEP3140}.
Otherwise, we could not distinguish \pythonil{l1} and \pythonil{l2} in the output.%
\end{sloppypar}%
%
Another good example of the difference between \pythonilIdx{str} and \pythonilIdx{repr} is \python's \pythonilIdx{datetime}\pythonIdx{datetime!datetime} class.
We will not discuss this class here in any detail.
It suffices to know that instances of this class represent a combination of a date and a time.
In the program, we first import the class \pythonilIdx{datetime} from the module of the same name.
We create a variable \pythonil{right\_now} and assign to it the result of the function \pythonilIdx{now}\pythonIdx{datetime!datetime!now}, which returns an object representing, well, today and the current time.\footnote{%
In the output of our program given in \cref{exec:dunder:str_vs_repr}, you cannot see the time of your reading, but the time when this book was compiled.}

If we want to print the result of the \pythonilIdx{str} function applied to an object~\pythonil{o} in an \pgls{fstring}, then we can either do this using the format specifier~\pythonil{!s}\pythonIdx{"!s} or by printing the result of \pythonil{str(o)}.
The former variant is usually preferred.
Anyway, we find that the simple string representation of a \pythonilIdx{datetime} object is, well, a simple human readable date and time string.
The result of the function \pythonilIdx{repr} for an object~\pythonil{o} can be obtained using the format specifier~\pythonil{!r}\pythonIdx{"!r} or by printing the result of \pythonil{repr(o)}.
Doing this with a \pythonilIdx{datetime} object gives us all the information that we need to manually recreate the object.
We could copy the output of \pythonilIdx{repr} from \cref{exec:dunder:str_vs_repr} into the \python\ console!
This would re-create the \pythonil{right\_now} object with the same data.
This would also work with the string representations that we printed for our lists \pythonil{l1} and \pythonil{l2} above.

\gitPythonAndOutput{\programmingWithPythonCodeRepo}{09_dunder}{point_user_2.py}{--args format}{dunder:point_user_2}{%
Investigating string representations and equality for the class~\pythonil{Point}.}%
%
Let us now move a bit backwards and revisit a previous example we created by ourselves.
In \cref{sec:immutableClassPoints2D}, we created the class \pythonil{Point} for representing points in the two-dimensional Euclidean plane~(see \cref{lst:classes:point}).
This class turned out to be quite useful when we went on to implement classes for different two-dimensional geometric shapes.
Here, we already implemented one dunder method, the initializer~\pythonilIdx{\_\_init\_\_}\pythonIdx{dunder!\_\_init\_\_}.
Let us play with this class a bit more.

In \cref{lst:dunder:point_user_2}, we create three instances of this class.
\pythonil{p1}~represents the coordinates~$(3,5)$, \pythonil{p2}~stores~$(7, 8)$, and~\pythonil{p3}~has the same coordinates as~\pythonil{p1}.
In this program, we first print the \pythonil{str} and \pythonil{repr} results for~\pythonil{p1}.
We immediately find them very unsatisfying.
Since we implemented neither \pythonilIdx{\_\_str\_\_}\pythonIdx{dunder!\_\_str\_\_} nor \pythonilIdx{\_\_repr\_\_}\pythonIdx{dunder!\_\_repr\_\_}, the default result for \pythonil{str} falls back to the result of \pythonil{repr} which then falls back to just the type name and object~ID.
This gives us basically no useful information.

While we are on the subject of \inQuotes{not useful,} there is another aspect of our \pythonil{Point} class that does not show useful behavior.
Way back in \cref{sec:equalityAndIdentity}, we discussed the difference between object identity and object equality.
All three variables~\pythonil{p1}, \pythonil{p2}, and~\pythonil{p3} point to different objects.
While \pythonil{p1 is p1}\pythonIdx{is} is obviously \pythonil{True}, \pythonil{p1 is p2} and \pythonil{p1 is p3} are obviously~\pythonil{False}.
The three objects are not all different instances of~\pythonil{Point}, so this is expected.

However, we find it annoying that \pythonil{p1 == p3}\pythonIdx{==} is \pythonil{False}, too.
\pythonil{p1 == p2} should be (and is) \pythonil{False}, because the two points are different.
But the two points \pythonil{p1} and \pythonil{p3} have the same coordinates.
They should be considered equal for all intents and purposes.
Vice versa, \pythonil{p1 != p2}\pythonIdx{"!=} should be (and is) \pythonil{True}, but \pythonil{p1 != p3}\pythonIdx{"!=} should be \pythonil{False} but turns out to be \pythonil{True}.

The reason for this is that \python\ cannot know when and why instances of our own class should be equal.
So it simply assumes that equality~$=$~identity, i.e., only identical instances are equal.
We could fix this by implementing the \pythonilIdx{\_\_eq\_\_}\pythonIdx{dunder!\_\_eq\_\_} dunder method.
This method would receive an arbitrary object~\pythonil{other} as input and should return \pythonil{True} if that is equal to the object whose method was invoked.

If you implement \pythonilIdx{\_\_eq\_\_}\pythonIdx{dunder!\_\_eq\_\_}, \python\ will make the reasonable assumption that \pythonil{(a != b) == not (a == b)}\pythonIdx{"!=}\pythonIdx{==}, i.e., assume that two objects are unequal if and only if they are not equal~\cite{PEP207}.
However, this is not necessarily always the case\footnote{%
In~\cite{PEP207}, it is stated that IEEE~754 floating point numbers do not satisfy that \pythonilIdx{==} and \pythonil{!=}\pythonIdx{"!=} are each other's complements. %
However, I could not find for an example where this was true in the standard~\cite{IEEE2019ISFFPA}, maybe with the exception of signaling~\pythonilsIdx{nan}, which does not matter in \python. %
Maybe it was true for some \python\ implementations back then, as~\cite{PEP754} indicates.%
}. %
Therefore, \python\ also allows us to implement an \pythonilIdx{\_\_ne\_\_}\pythonIdx{dunder!\_\_ne\_\_} dunder method to realize inequality differently or, potentially, more efficiently, instead~\cite{PEP207}.

Finally, we compare whether \pythonil{p1} is the same as the integer number~\pythonil{5}.
This, obviously, should return~\pythonil{False}.
And it does so.
This is because the two objects~\pythonil{p1} and~\pythonil{5} are not identical.
The default equality comparison only checks for identity.
If implement \pythonilIdx{\_\_eq\_\_} by ourselves, this method should clearly return a value that makes \pythonil{p1 == 5} become~\pythonil{False} as well.
Anything else would be nonsense.

\gitPython{\programmingWithPythonCodeRepo}{09_dunder/point_with_dunder.py}{--args format}{dunder:point_with_dunder}{%
Our \pythonil{Point} class, extended with the \pythonilIdx{\_\_str\_\_}\pythonIdx{dunder!\_\_str\_\_}, \pythonilIdx{\_\_repr\_\_}\pythonIdx{dunder!\_\_repr\_\_}, and \pythonilIdx{\_\_eq\_\_}\pythonIdx{dunder!\_\_eq\_\_}\pythonIdx{NotImplementedType} dunder methods.}%
%
\gitPythonAndOutput{\programmingWithPythonCodeRepo}{09_dunder}{point_with_dunder_user.py}{--args format}{dunder:point_with_dunder_user}{%
The same program exploring string representations and equality as shown in \cref{lst:dunder:point_user_2}, but this time using our new \pythonil{Point} class from \cref{lst:dunder:point_with_dunder}.}%

In order to fix all of the problems discussed above, we implement the three dunder methods \pythonilIdx{\_\_str\_\_}\pythonIdx{dunder!\_\_str\_\_}, \pythonilIdx{\_\_repr\_\_}\pythonIdx{dunder!\_\_repr\_\_}, and \pythonilIdx{\_\_eq\_\_}\pythonIdx{dunder!\_\_eq\_\_} for our \pythonil{Point} class in \cref{lst:dunder:point_with_dunder}.
The concise string representation returned by \pythonilIdx{\_\_str\_\_}\pythonIdx{dunder!\_\_str\_\_} will just be the point coordinates in parentheses.
This offers all the information needed at a glance, but it could be mistaken with a tuple as string.
Therefore, the canonical string representation produced by \pythonilIdx{\_\_repr\_\_}\pythonIdx{dunder!\_\_repr\_\_} will return a string of the shape~\pythonil{"Point(x, y)"}.

Finally, the \pythonilIdx{\_\_eq\_\_}\pythonIdx{dunder!\_\_eq\_\_} method will first check if the \pythonil{other} object is an instance of \pythonil{Point}.
If so, it will return \pythonil{True} if and only if the \pythonil{x} and \pythonil{y} coordinate of the \pythonil{other} point are the same as of the point \pythonil{self}.
Otherwise, it will return the constant \pythonilIdx{NotImplemented}:%
%
\quotation{PSF2024BIC}{%
A special value which should be returned by the binary special methods [\dots] to indicate that the operation is not implemented with respect to the other type\dots\medskip\\\strut\hspace{1cm}\strut%
\emph{Note:}~When a binary (or in-place) method returns \pythonilIdx{NotImplemented} the interpreter will try the reflected operation on the other type (or some other fallback, depending on the operator). %
If all attempts return \pythonilIdx{NotImplemented}, the interpreter will raise an appropriate exception. %
Incorrectly returning \pythonilIdx{NotImplemented} will result in a misleading error message or the \pythonilIdx{NotImplemented} value being returned to \python\ code.
}%
%
In other words, our \pythonilIdx{\_\_eq\_\_} method can only compare the current~\pythonil{Point} for equality with another~\pythonil{Point}.
If \pythonil{other} is not an instance of~\pythonil{Point}, then no way to compare for equality with it exists.
Now, we could return \pythonil{False} in this case, which would be fine as well.
Returning \pythonilIdx{NotImplemented} will give us the same result in comparisons with objects of other types~(like~\pythonil{5}).
However, it keeps an avenue open for other programmers to design new classes which support comparison with our \pythonil{Point} instances in a consistent way.
When we implement the \pythonilIdx{\_\_eq\_\_} method like this, the proper \pgls{typeHint} for the return value is \pythonil{bool | NotImplementedType}\pythonIdx{NotImplementedType}.

\Cref{lst:dunder:point_with_dunder_user} is the same as \cref{lst:dunder:point_user_2}, but now uses this new variant of our class \pythonil{Point}.
As you can see in \cref{exec:dunder:point_with_dunder_user}, its output now matches much better to what one would expect.%
%
\FloatBarrier%
\endhsection%
%
%
\hsection{\texttt{\_\_hash\_\_}, \texttt{\_\_eq\_\_} and \texttt{dict} and \texttt{set}}%
\label{sec:hashDunder}%
%
\gitPython{\programmingWithPythonCodeRepo}{09_dunder/point_with_hash.py}{--args format}{dunder:point_with_hash}{%
Our \pythonil{Point} class, extended with the \pythonilIdx{\_\_eq\_\_}\pythonIdx{dunder!\_\_eq\_\_} and \pythonilIdx{\_\_hash\_\_}\pythonIdx{dunder!\_\_hash\_\_} dunder methods.}%
%
\gitPythonAndOutput{\programmingWithPythonCodeRepo}{09_dunder}{point_with_hash_user.py}{--args format}{dunder:point_with_hash_user}{%
Using the new \pythonil{Point} class from \cref{lst:dunder:point_with_hash} in sets and dictionaries.}%
%
Our \pythonil{Point} objects are immutable and just consist of two numerical coordinates.
Maybe there could be an application where someone would like to use them as keys for a dictionary or would like to construct a set of points.
For this to be possible, two things are needed:%
%
\begin{itemize}%
%
\item A dunder method \pythonilIdx{\_\_eq\_\_} that compares two \pythonils{Point} for equality. %
This, we already have done in \cref{lst:dunder:point_with_dunder}.%
%
\item A dunder method \pythonilIdx{\_\_hash\_\_} that returns the hash value of a \pythonil{Point} in form of an~\pythonil{int}.%
%
\end{itemize}%
%
For these two methods, it must hold that~\cite{PSF2024OH}%
%
\begin{equation}%
\pythonil{a.\_\_eq\_\_(b)} \Rightarrow \pythonil{a.\_\_hash\_\_()} = \pythonil{b.\_\_hash\_\_()}%
\label{eq:eqAndHash1}%
\end{equation}%
%
This is equivalent~\cite{PSF2024BIE,PSF2024OH} to:%
%
\begin{equation}%
\pythonil{a == b} \Rightarrow \pythonil{hash(a)} = \pythonil{hash(b)}%
\label{eq:eqAndHash2}%
\end{equation}%
%
But let us step back a bit here.
What is a hash value?
Why is an integer hash value needed?
Why does the equality of two objects require them to have the same hash value?

Dictionaries in \python\ (and \pgls{Java}) internally use tables, where key-value relationships are stored~\cite{G2020PHTUTH,L2011PDI}.
Sets are basically the same, but only store the keys.
The internal tables could be represented as lists.
Differently from lists, new elements are not added at the end.
Instead, they would be more like lists of a fixed length where new elements are placed at specific indices where they can be found again.
These hash tables~\cite{K1998SAS,CLRS2009ITA,SKS2019DSC} are very fast.
They have an element-wise access/update time complexity of~\bigOb{1}~\cite{H2024PBOTTCODDSIPP33,B2023T,N2022CCSFPO}.
As you know, lists can be indexed by integers.
Since we also want to be able to use objects that are \emph{not} integers as keys, too, we need a way to \inQuotes{translate} these objects to integers.
That is what \pythonilIdx{\_\_hash\_\_} is supposed to do.

Luckily, \pythonilIdx{\_\_hash\_\_} does not need to compute a valid index into such a table.
That would indeed not be possible:
The \pythonilIdx{\_\_hash\_\_} method of an object~\pythonil{a} cannot know the length of the internal table underlying a dictionary or set where the object~\pythonil{a} will be stored.
Instead, all it needs to do is to return one integer value that represents~\pythonil{a}.
The dictionary/set implementation is then responsible for translating this integer into a valid index into its internal table.
It can use \pgls{modulodiv} for this.
It will also do lots of other things, e.g., taking care of collisions, i.e., cases where two different objects have the same hashes (or different hashes that end up mapping to the same index)~\cite{G2020PHTUTH,L2011PDI}.
None of this is important here.

What is important is that the hash values are needed to find the objects in the dictionaries and set.
We want to know whether the object~\pythonil{a} is \pythonilIdx{in} the set~\pythonil{s}?
Then the set~\pythonil{s} uses \pythonil{hash(a)}\pythonIdx{hash} which invokes~\pythonil{a.__hash__()}\pythonIdx{\_\_hash\_\_} to get the hash value of~\pythonil{a}.\footnote{%
This works the same way in which \pythonil{repr(a)} would invoke \pythonil{a.__repr__()} if it is defined.}
It translates the hash value to an index.
It checks its internal table whether there is an object~\pythonil{b} at that index with~\pythonil{b == a}.
If yes, then, well, \pythonil{a} is in the set~\pythonil{s}, i.e., \pythonil{a in s} yields~\pythonil{True}.
If not, then not.
As said, the reality is more complex, because of potentially occurring collisions, but for our excursion here, this very coarse approximation of how this works shall suffice.

Anyway, if we want to add an object to the set~\pythonil{s}:
Then again the index is computed via the hash value and if the object is not already there, it is placed there.
Dictionaries work the same, just use key-value relationships, where the hash values of the keys are computed to find the right places in the internal tables.
The details are (for the third time), unimportant here -- but you can read them in these very interesting sources here~\cite{G2020PHTUTH,L2011PDI}.

It is already clear, however, that calling \pythonilIdx{\_\_hash\_\_} twice for the same object~\pythonil{a} must return the same value.
Since this hash value is used to find the place in the table where the object should be, this must never change.
This is also why dictionary (and set) keys must be immutable~(see~\cref{bp:dictKeys,bp:setsImmutable}).

It is also clear that two objects~\pythonil{a} and~\pythonil{b} that are \emph{equal} must also have the same hash value.
If two objects are equal, it should be that $\pythonil{a in s}=\pythonil{b in s}$.
Otherwise, it could be that \pythonil{"123" in s} is \pythonil{True} for the string constant~\pythonil{"123"}, but if we read the string~\pythonil{"123"} from a file, \pythonil{"123" in s} could return~\pythonil{False}.
That would make no sense at all.

With that out of the way, we can now make our \pythonil{Point} class hashable.
In \cref{lst:dunder:point_with_hash}, we modify the \pythonil{Point} class from \cref{lst:dunder:point_with_dunder}.
We retain the implementation of~\pythonilIdx{\_\_eq\_\_} and add the method~\pythonilIdx{\_\_hash\_\_}.%
%
\quotation{PSF2024OH}{{\dots}The \pythonil{\_\_hash\_\_()}\pythonIdx{\_\_hash\_\_}\pythonIdx{dunder!\_\_hash\_\_} method should return an integer. %
The only required property is that objects which compare equal have the same hash value; %
it is advised to mix together the hash values of the components of the object that also play a part in comparison of objects by packing them into a \pythonil{tuple} and hashing the tuple.}%
%
\bestPractice{hash}{%
For implementing \pythonilIdx{\_\_eq\_\_} and \pythonilIdx{\_\_hash\_\_}, the following rules hold~\cite{PSF2024OH}:%
\sloppy%
\begin{itemize}\sloppy%
%
\item Only immutable classes are allowed to implement \pythonilIdx{\_\_hash\_\_}, i.e., only classes where all attributes have the \pythonilIdx{Final} \pgls{typeHint} and are only assigned on the initialize~\pythonilIdx{\_\_init\_\_}.%
%
\item The result of \pythonil{a.\_\_hash\_\_()} must never change (since \pythonil{a} must never change either).%
%
\item If a class does not define \pythonilIdx{\_\_eq\_\_}, it cannot implement \pythonilIdx{\_\_hash\_\_} either.%
%
\item Instances of a class that implements \pythonilIdx{\_\_eq\_\_} but not \pythonilIdx{\_\_hash\_\_} cannot be used as keys in a dictionary or set.%
%
\item Only instances of a class that implements both \pythonilIdx{\_\_eq\_\_} and \pythonilIdx{\_\_hash\_\_} can be used as keys in dictionaries or sets.%
%
\item Then, the results of \pythonilIdx{\_\_eq\_\_} and \pythonilIdx{\_\_hash\_\_} must be computed using the exactly same attributes. %
In other words, the attributes of an object~\pythonil{a} that determine the results of~\pythonil{a.\_\_eq\_\_(...)} must be exactly the same as those determining the results of~\pythonil{a.\_\_hash\_\_(...)}. %
%
\item It is best to compute \pythonil{a.\_\_hash\_\_(...)} by simply putting all of these attributes into a \pythonil{tuple} and then passing this \pythonil{tuple} to \pythonilIdx{hash}.%
%
\item Two objects that are equal must have the same hash value, i.e., \cref{eq:eqAndHash1,eq:eqAndHash2} must hold.%
\end{itemize}%
\fussy%
}%
%
That sounds complicated, but is actually very easy.
The only attributes that play a role in our \pythonilIdx{\_\_eq\_\_} method are the two coordinates of the point, \pythonil{self.x} and~\pythonil{self.y}.
So the result~\pythonilIdx{\_\_hash\_\_} should simply be~\pythonil{hash((self.x, self.y))}\pythonIdx{hash}.
The double-parentheses are because this basically means \pythonil{t = (self.x, self.y)} and then computing~\pythonil{hash(t)}.%
%
\begin{sloppypar}%
One may be a bit scared regarding integers and floats.
We permitted the coordinates of our points to be either \pythonils{int} or \pythonils{float}.
Since \pythonil{5.0 == 5} holds, we could feel anxious whether \pythonil{hash(5.0) == hash(5)}\pythonIdx{hash} would also hold.
If \emph{not}, then something like \pythonil{hash((5.0, 3)) != hash((5, 3))} could happen, which could lead to~\pythonil{Point(5.0, 3).\_\_hash\_\_() != Point(5, 3).\_\_hash\_\_()} while \pythonil{Point(5.0, 3).\_\_eq\_\_(Point(5, 3))} is~\pythonil{True}.
This would then violate~\cref{bp:hash}.%
\end{sloppypar}%
%
If that would happen, we could create sets that contain equal elements multiple times.
Which, in turn, would violate the definition of sets.
That would be a very counter-intuitive bug in our code.
Actually, I secretly planned to use this as a very tricky example for learning how to use the debugger{\dots}
Alas, the developers of \python\ have already solved this:%
%
\quotation{PSF2024BIF}{Numeric values that compare equal have the same \pythonilIdx{hash} value (even if they are of different types, as is the case for~\pythonil{1} and~\pythonil{1.0}).}%
%
Therefore, we can indeed implement \pythonilIdx{\_\_hash\_\_} with a single line of code in~\cref{sec:hashDunder}.
And I will later find another example on how the debugger can be used to spot errors in code.

We use our new variant of the \pythonil{Point} class in~\cref{lst:dunder:point_with_hash_user}.
We again first create three points~\pythonil{p1 = Point(3, 5)}, \pythonil{p2 = Point(7, 8)} and \pythonil{p3 = Point(3, 5.0)}.
\pythonil{p1 == p2} is \pythonil{False}, while \pythonil{p1 == p3} is \pythonil{True}, despite the \pythonil{y}\nobreakdashes-coordinate of \pythonil{p1} is an~\pythonil{int} and the one of~\pythonil{p2} is a~\pythonil{float}~(but with the same value).
When we create the set~\pythonil{points} as~\pythonil{\{p1, p2, p3\}}, it will have the size~2.
Since \pythonil{p1 == p3}, only one of these two objects is stored in the set.
However, \pythonil{p1 in points}, \pythonil{p2 in points}, and~\pythonil{p3 in points} are all~\pythonil{True}.
This is because \pythonil{p1} and \pythonil{p3} also have the same hash value.

If we create a new point~\pythonil{p4} with coordinates equal to those of~\pythonil{p2}, then \pythonil{p4 in points} will also hold.
However, a point~\pythonil{p5} whose coordinates are different from those of~\pythonil{p1} and~\pythonil{p2} will not be an element of~\pythonil{points}, i.e., \pythonil{p5 in points} would be~\pythonil{False}.

Now we can also use the instances of our class~\pythonil{Point} as keys for a dictionary~\pythonil{point_vals}.
The same dictionary operations as discussed way back in~\cref{sec:dictionaries} can be used without problems.%
%
\FloatBarrier%
\endhsection%
%
\endhsection%
