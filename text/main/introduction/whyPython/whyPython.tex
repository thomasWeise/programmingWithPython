\hsection{Why Python?}%
%
The center of this course is the \python\ programming language.
Our goal is to get familiar with programming, with the programming language \python, and with the tools and ecosystem surrounding it.
This makes sense for several reasons.%
%
\begin{figure}%
\centering%
\includegraphics[width=0.75\linewidth]{\currentDir/languagesByGithubPushes}%
\caption{The twelve most popular programming languages chosen based on the \github\ pushes over the years. Source:~\cite{B2023G2GLS}.}%
\label{fig:languagesByGithubPushes}%
\end{figure}%

First, \python\ is one of the most successful and widely used programming languages~\cite{CBST2024LOHPPTDDSAMLA}.
We plot the number of pushes to \github\ over time for the most popular programming and web development languages in \cref{fig:languagesByGithubPushes}.
We find that \python\ became the leading languages at some point in 2018.
\python\ is intensely used in the fields of \pgls{AI}, \pgls{ML}, and \pgls{DS}~\cite{CBST2024LOHPPTDDSAMLA} as well as optimization, which are among the most important areas of future technology.
If you will do programming in any future employment or research position, chances are that \python\ knowledge will be useful.

Second, there exists a very large set of powerful libraries supporting both research and application development in these fields, including \numpy~\cite{HMvdWGVCWTBSKPHvKBHFdRWPGMSRWAGO2020APWN}, \pandas~\cite{B2012DPWP}, \scikitlearn~\cite{PVGMTGBPWDVPCBPD2011SMLIP}, \scipy~\cite{VGOHRCBPWBvdWBWMMNJKLCPFMVLPCHQHARPvMS2020SFAFSCIP}, \tensorflow~\cite{ABCCDDDGIIKLMMMSTVWWYZ2016TASFLSML}, \matplotlib~\cite{H2007MA2GE}, and \moptipy~\cite{WW2023RSDEWASSAA}\footnote{Yes, I list \moptipy\ here, next to very well-known and widely-used frameworks, because I am its developer.}, just to name a few.
This means that for many tasks, you can find suitable and efficient \python\ libraries that support your work.

Third, \python\ is very easy to learn~\cite{GPBS2006WCTIPIHSUP,VR1999CPFERPASEFTPOT}.
It has a simple and clean syntax and enforces a readable structure of programs.
Programmers do not need to declare data types explicitly\footnote{at least during the first steps of learning}.
\python\ has expressive built-in types likes lists, tuples, and dictionaries.
%
\begin{figure}%
\centering%
\includegraphics[width=0.7\linewidth]{\currentDir/pythonIsInterpreted.pdf}%
\caption{\python\ code is interpreted, which leads to an easier programming workflow compared to compiled programming languages like \softwareStyle{C}.}%
\label{fig:pythonIsInterpreted}%
\end{figure}%
%
The fact that \python\ is an interpreted language makes it somewhat slower compared to compiled languages like \softwareStyle{C}.
However, this also leads to a much easier workflow when experimenting and programming, as sketched in \cref{fig:pythonIsInterpreted}.
It also is possible to interactively write programs in an interpreter window.
This means that you can execute commands in a \gls{terminal} instead of needing to compile and run programs.
These features, in sum, make \python\ a good choice for learning how to write programs.%
%
\endhsection%
%
